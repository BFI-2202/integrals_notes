\documentclass{article}

\usepackage[T2A]{fontenc}
\usepackage[utf8]{inputenc}
\usepackage[russian]{babel}

\usepackage{tabularx}
\usepackage{amsmath}
\usepackage{pgfplots}
\usepackage{geometry}
\usepackage{multicol}
\geometry{
    left=1cm,right=1cm,top=2cm,bottom=2cm
}
\newcommand*\diff{\mathop{}\!\mathrm{d}}

\newtheorem{definition}{Определение}
\newtheorem{theorem}{Теорема}

\DeclareMathOperator{\sign}{sign}

\usepackage{hyperref}
\hypersetup{
    colorlinks, citecolor=black, filecolor=black, linkcolor=black, urlcolor=black
}

\title{Интегралы и дифференциальные уравнения}
\author{Лисид Лаконский}
\date{October 2023}

\begin{document}
\raggedright

\maketitle

\tableofcontents
\pagebreak

\section{Лекция — 17.10.2023}

\subsection{Дифференциальные уравнения}

\begin{definition}
    \textbf{Дифференциальным уравнением} называется уравнение, связывающее независимую переменную $x$, искомую функцию $y$ и ее какие-то производные:
    $$
    F(x, y, y', y'', \dots) = 0
    $$
    \textbf{Обыкновенное дифференциальное уравнение}:
    $$
    y' = f(x)
    $$
    \textbf{Дифферециальное уравнение в частных производных}:
    $$
    y' = f(x, t)
    $$
\end{definition}

\begin{definition}
    \textbf{Порядком} в дифференциальном уравнении называется порядок наивысшей производной, входящей в это уравнение.
\end{definition}

\begin{definition}
    \textbf{Решением} (интегралом) дифференциального уравнения называется всякая функция, которая будучи подставлена в уравнение обращает его в верное равенство.
\end{definition}

\paragraph{Пример №1}

Пусть

$$y'' - y = 0$$

Проверим:

\begin{enumerate}
    \item $y = x$

    $y' = 1$, $y'' = 0$

    Таким образом, не является решением.
    \item $y = \sin x$

    $y' = \cos x$, $y'' = -\sin x$

    Таким образом, не является решением
    \item $y = e^{x}$

    $y' = e^{x}$, $y'' = e^{x}$

    Является решением.
\end{enumerate}

\begin{definition}
    \textbf{Общим решением} дифференциального уравнения называется функция $y = \phi(x; C)$ ($C = const$), которая
    
    \begin{enumerate}
        \item удовлетворяет дифференциальному уравнению при любом значении $C$;
        \item каково бы не было начальное условие, всегда можно подобрать значение $C_0$, чтобы оно удовлетворяло указанному начальному условию.
    \end{enumerate}
\end{definition}

\paragraph{Пример №1}

$y = x^2 + C$

$y(0) = 4$

$y = x^2 + 4$ — Решение задачи Коши, удовлетворяет нач. усл.

\paragraph{Пример №2}

Найти уравнение кривой, у которой точка пересечения любой касательной с осью $OX$ $K$ равноудалена от нач. координат $O$ и от точки касания $M(x, y)$.

Условие, которое должно быть выполнено: $|OK| = |OM|$

Уравнение касательной:

$Y - y = y' (X - x)$

В т.к $Y = 0$, получаем:

$-y = y'(X - x) \Leftrightarrow x - \frac{y}{y'} = X$. Таким образом, можем теперь задать коордианты т. K: $(x - \frac{y}{y'}; 0)$

$|OK| = |x - \frac{y}{y'}|$

$|KM| = \sqrt{(x - (x - \frac{y}{y'}))^2 + (y - 0)^2}$

$|x - \frac{y}{y'}| = \sqrt{(\frac{y}{y'})^2 + y^2}$

$x^2 - \frac{2xy}{y'} + (\frac{y}{y'})^2 = (\frac{y}{y'})^2 + y^2$

$-\frac{2xy}{y'} = -y^2 + x^2$

$y' = \frac{2xy}{x^2-y^2}$ — Дифференциальное уравнение (I порядок)

\paragraph{Геометрическая интерпретация решения дифференциального уравнения}

Общее решение (общий интеграл) представляет собой семейство кривых на координатной плоскости, а решение задачи Коши представляет собой кривую, проходящую через заданную точку.

\subsubsection{Теорема о существовании и единственности решения}

\begin{theorem}[О существовании и единственности решения]
    Пусть в дифференциальном уравнении $y' = F(x, y)$ функция $F(x, y)$ и ее частная производная $\frac{\delta f}{\delta y}$ непрерывны на открытом множестве $G$. В этом случае
    \begin{enumerate}
        \item Для всякой точки $(x_0, y_0) \in G$ найдется решение $y = y(x)$ дифференциального уравнения $y' = f(x, y)$, удовлетворяющее условию $y_0 = y(x_0)$ (решение задачи Коши).
        \item Если два решения ($y = y_1(x)$, $y = y_2(x)$) дифференциального уравнения $y' = F(x, y)$ совпадают хотя бы для одного решения $x^{*}$. то эти решения совпадают для всех тех значений переменной $x$, для которой они определены.
    \end{enumerate}
\end{theorem}

\subsubsection{Дифференциальный уравнения (I порядок) с разделенными или разделяющимися переменными}

Дифференциальное уравнение (I порядок) может быть записано:

\begin{enumerate}
    \item $y' = f(x; y)$
    \item $p(x; y) \diff x + q(x, y) \diff y = 0$
\end{enumerate}

\begin{definition}
    \textbf{Уравнением с разделяющимися переменными} называют дифференциальное уравнение, в котором функция $f$ может быть разбита на две такие функции, разделенные знаками умножения или деления, что одна из них зависит только от $x$, а другая зависит только от $x$.

    $$y' = f_1(x) * f_2(y) \ \ \textbf{(1)}$$
    $$p_1(x) * p_2(y) + q_1(x) * q_2(y) \diff y = 0 \ \ \textbf{(2)}$$
\end{definition}

Разделение переменных:

\begin{enumerate}
    \item $\frac{\diff y}{\diff x} = f_1(x) * f_2(y)$

    $\frac{\diff y}{f_2(y)} = f_1(x) \diff x$
    \item $p_1(x) p_2(y) \diff x = -q_1(x) q_2(y) \diff y$

    $\frac{p_1(x) \diff x}{q_1(x)} = -\frac{q_2(y) \diff y}{p_2(y)}$
\end{enumerate}

После этого мы можем интегрировать левую и правую часть:

\begin{enumerate}
    \item $\int \frac{\diff y}{f_2(y)} = \int f_1(x) \diff x$
    \item $\int \frac{p_1(x) \diff x}{q_1(x)} = -\int \frac{q_2(y) \diff y}{p_2(y)}$
\end{enumerate}

Получим:

\begin{enumerate}
    \item $\Phi_2(y) = \Phi_1(x) + C$

    или

    $\Phi_2(y) - \Phi_1(x) = C$
    \item $F_1(x) = F_2(y) + C$
\end{enumerate}

\paragraph{Пример №1}

$\sqrt{y^2 + 1} \diff x = xy \diff y$

Выполним разделение переменных:

$\frac{\diff x}{x} = \frac{y \diff y}{\sqrt{y^2 + 1}}$

Интегрируем полученное:

$\int \frac{\diff x}{x} = \int \frac{y \diff y}{\sqrt{y^2 + 1}}$

$\ln |x| = (y^2 + 1)^{-\frac{1}{2}} (y^2 + 1)$

$\ln |x| = \frac{1}{2} \frac{(y^2 + 1)^{1/2}}{1/2} + C$

$\ln |x| = (y^2 + 1)^{1/2} + C$

\textbf{Ответ}: $\ln Cx = \sqrt{y^2 + 1}$

\paragraph{Замечание о приведении к уравнениям с разделющимися переменными}

Уравнение вида $y' = f(ax + by + c)$ можно привести заменой $z = ax + by$ или $ax + by + C$ к уравнению с разделяющимися переменными.

\paragraph{Пример №2}

$(x + 2y) y' = 1$

$y' = \frac{1}{x + 2y}$

Заменим: $z = x + 2y$

$\frac{z - x}{2} = y$

$\frac{z' - 1}{2} = y'$

$\frac{z' - 1}{2} = \frac{1}{z}$

$\frac{\diff z}{\diff x} = z' = \frac{2}{z} + 1 = \frac{2 + z}{z}$

$\frac{z}{z + 2} \diff z = \diff x$

Переменные теперь разделены и их можно проинтегрировать:

$\int \frac{z + 2 - 2}{z + 2} \diff z = \int \diff x$

$\int (1 - \frac{2}{z + 2}) \diff z = \int \diff x$

$\int \frac{\diff z}{z + 2} = \ln |z + 2|$, $\int \diff z = 2$

$z - 2 \ln |z + 2| = x + C$

$x + 2y - 2 \ln |x + 2y + 2| = x + C$

\textbf{Ответ:} $2y - 2 \ln |x + 2y + 2| = C$

\subsubsection{Однородные дифференциальные уравнения (I порядок)}

\begin{definition}
    \textbf{Функция $f(x y)$ называется однородной функцией} $n$-го измерения относительно переменных $x$ и $y$, если для любой $\lambda$ справедливо следующее:
    $$
    f(\lambda x, \lambda y) = \lambda^{n} f(x; y)
    $$
\end{definition}

Пример однородной функции второго измерения:

$$f(x, y) = \sqrt[3]{x^6 + y^6}$$

$$f(\lambda x, \lambda y) = \sqrt[3]{\lambda^6 x^6 + \lambda^6 y^6} = \sqrt[3]{\lambda^6 (x^6 + y^6)} = \lambda^2 \sqrt{x^6 + y^6}$$

Пример функции, не являющейся однородной:

$$f(x, y) = \sqrt[3]{x^6 + y^3}$$

$$f(\lambda x, \lambda y) = \sqrt[3]{\lambda^6 x^6 + \lambda^3 y^3} = \lambda^3 \sqrt{\lambda^3 x^3 + y^6}$$

\begin{definition}
    \textbf{Дифференциальное уравнение (I порядок) называется однородным} относительно $x$ и $y$, если $f(x, y)$ является однородной функцией нулевого измерения:

    $$f(\lambda x, \lambda y) = \lambda^{0} f(x, y)$$
\end{definition}

Пример однородного дифференциального уравнения:

$$f(x, y) = \frac{x^2 - y^2}{2xy}$$

$$f(\lambda x, \lambda y) = \frac{\lambda^2 x^2 - \lambda^2 y^2}{2 \lambda x \lambda y} = \frac{x^2 - y^2}{2xy}$$

\paragraph{Решение однородных дифференциальных уравнений (I порядок)} 

Необходимо заменить:

$$t = \frac{y}{x}$$

$$y = tx$$

$$y' = t' x + t$$

Получим:

$$t' * x + t = f(t)$$

$$t' * x = f(t) - t$$

$$\frac{\diff t}{\diff x} * x = f(t) - t$$

$$\int \frac{\diff t}{f(t) - t} = \int \frac{\diff x}{x}$$

Переменные разделились и мы интегрируем, не забывая обратно заменить $t$.

\paragraph{Пример №1}

$y' = \frac{x + 2y}{x}$

$y' = 1 + 2 * \frac{y}{x}$

Заменим:

$t = \frac{y}{x}$, $y = tx$, $y' = t' x + t$

Получим:

$t'x + t = 1 + 2t$

$t' * x = 1 + t$

$\frac{\diff t}{\diff x} * x = 1 + t$

Выполним окончательное разделение:

$\frac{\diff t}{t + 1} = \frac{\diff x}{x}$

Теперь мы можем интегрировать:

$\int \frac{\diff t}{t + 1} = \int \frac{\diff x}{x}$

$\ln |t + 1| = \ln |x| + \ln C$

$\ln |t + 1| = \ln Cx$

$t + 1 = Cx$

$\frac{y}{x} = Cx - 1$

\textbf{Ответ:} $y = Cx^2 - x$


\end{document}