\documentclass{article}

\usepackage[T2A]{fontenc}
\usepackage[utf8]{inputenc}
\usepackage[russian]{babel}

\usepackage{tabularx}
\usepackage{amsmath}
\usepackage{pgfplots}
\usepackage{geometry}
\usepackage{multicol}
\geometry{
    left=1cm,right=1cm,top=2cm,bottom=2cm
}
\newcommand*\diff{\mathop{}\!\mathrm{d}}

\newtheorem{definition}{Определение}
\newtheorem{theorem}{Теорема}

\DeclareMathOperator{\sign}{sign}

\usepackage{hyperref}
\hypersetup{
    colorlinks, citecolor=black, filecolor=black, linkcolor=black, urlcolor=black
}

\title{Интегралы и дифференциальные уравнения}
\author{Лисид Лаконский}
\date{October 2023}

\begin{document}
\raggedright

\maketitle

\tableofcontents
\pagebreak

\section{Практическое занятие — 03.10.2023}

\subsection{Криволинейные интегралы (II род)}

На прошлой паре мы записывали:

$\int\limits_{M}^{N} P(x y) \diff x + Q(x, y) \diff y$ (по пути $L$) $= \int\limits_{t_1}^{t_2} P(x(t), y(t)) x'_t + Q(x(t), y(t) y'_t) \diff t$, $\begin{cases}
    x = x(t) \ \diff x = x' \diff t \\
    y = y(t) \ \diff y = y' \diff t \\
    t_1 \le t \le t_2
\end{cases}$

Но можно иначе:

$\begin{cases}
    x = x \ \diff x = \diff x \\
    y = y(x) \ \diff y = y' \diff x \\
    x_1 \le x \le x_2
\end{cases}, \int\limits_{M(x_1, y_1)}^{N(x_2, y_2)} P(x, y) \diff x + Q(x, y) \diff y = \int\limits_{x_1}^{x_2} (P(x, y(x)) + Q(x, y(x)) y'_x) \diff x$ по $L: y = y(x)$

\subsubsection{Условие независимости криволинейных интегралов от пути интегрирования}

\textbf{Если под знаком интеграла находится полный дифференциал какой-то функции, то ответ не будет зависеть от пути интегрирования, а только от начальной и конечной точки}.

$\int\limits_{(x_1, y_1)}^{(x_2, y_2)} \diff u = u \bigg|_{(x_1, y_1)}^{(x_2, y_2)} = u(x_2, y_2) - u(x_1. y_1)$

\textbf{Во всех других случаях криволинейный интеграл зависит от пути интегрирования.}

\textbf{Полный дифференциал} функции двух переменных:

$\diff u = \frac{\delta u}{\delta x} \diff x + \frac{\delta u}{\delta y} \diff y$

При этом:

$\frac{\delta}{\delta y} (\frac{\delta u}{\delta x}) = \frac{\delta}{\delta x} (\frac{\delta u}{\delta y})$

\begin{theorem}
    Если во всех точках области $G$ $P(x, y)$, $Q(x, y)$ частные производные $\frac{\delta P}{\delta y}$, $\frac{\delta Q}{\delta x}$ являются непрерывны, то необходимым и достаточным условием того, что криволинейный интеграл не зависит от пути интегрирования является выполнение условия $\frac{\delta P}{\delta y} = \frac{\delta Q}{\delta x}$ во всех точках области $G$.
\end{theorem}

\begin{theorem}
    При выполнении $\frac{\delta P}{\delta y} = \frac{\delta Q}{\delta x}$ криволиненый интеграл по замкнутому контуру равен нулю: $\oint (P \diff x + Q \diff y) = 0$
\end{theorem}

\subsubsection{Формула Грина–Остроградского}

$\oint\limits_{C} P(x, y) \diff x + Q(x, y) \diff y = \int\int\limits_{D} (\frac{\delta Q}{\delta x} - \frac{\delta P}{\delta y}) \diff x \diff y = \int\limits_{a}^{b} \diff x \int\limits_{y_1(x)}^{y_2(x)} \frac{\delta P}{\delta y} \diff y = \int\limits_{a}^{b} (P(x, y_2(x)) - P(x, y_1(x))) \diff x$, где $D$ — замкнутая область, ограниченная контуром $C$.

\textbf{Примечание:} функции $P(x, y)$, $Q(x, y)$ должны быть определены и непрерывны в области $D$ и, кроме того, иметь в ней непрерывные частные производные $\frac{\delta P}{\delta y}$, $\frac{\delta Q}{\delta x}$.

\paragraph{Следствие из данной формулы} Если $P = 0$, $Q = x$, то $\oint x \diff y = \int\int \diff x \diff y$ — площадь, ограниченная областью.

\subsubsection{Примеры}

\paragraph{Пример №1}

$\int\limits_{(1, 1)}^{(2, 2)} \frac{y \diff x + x \diff y}{x^2 + y^2}$ (по прямой $y = x$) $= \int \frac{y \diff x}{x^2 + y^2} + \frac{x \diff y}{x^2 + y^2} = \int\limits_{1}^{2} (\frac{x}{x^2 + x^2} + \frac{x}{x^2 + x^2}) \diff x = \int\limits_{1}^{2} \frac{2x \diff x}{2x^2} = \ln |x| \bigg|_{1}^{2} = \ln 2$

\paragraph{Пример №2}

$\int\limits_{(0, 0)}^{(1, 1)} 2x^2 y \diff x + x \sqrt{y} \diff y$. Вычислить данный интеграл по:

\begin{enumerate}
    \item прямой $y = x$;
    
    $\int\limits_{0}^{1} (2x^2 x + x \sqrt{x}) \diff x = \int\limits_{0}^{1} (2x^3 + x^{3/2}) \diff x = \frac{2x^4}{4/2} + \frac{2x^{5/2}}{5} \bigg|_{0}^{1} = \frac{9}{10}$
    \item параболе $y = x^2$;
    
    $\int\limits_{0}^{1} (2x^2 x^2 + x \sqrt{x^2} 2x) \diff x = \int\limits_{0}^{1} (2x^4 + 2x^3) \diff x = (\frac{2x^{5}}{5} + \frac{2x^{4}}{4}) \bigg|_{0}^{1} = \frac{9}{10}$
    \item ломанной линии, образованной $y = x$, $y = x^2$
    
    $\int\limits_{0}^{B} + \int\limits_{B}^{A} = \int\limits_{0}^{1} \sqrt{y} \diff y = \frac{2y^{3/2}}{3} \bigg|_{0}^{1} = \frac{2}{3}$
\end{enumerate}

\paragraph{Пример №3}

Пусть $P = (2x \cos y - y^2 \sin x)$, $Q = (2y \cos x - x^2 \sin y)$. Проверить независимость от пути интегрирования:

$\frac{\delta P}{\delta y} = -2x \sin y - 2 y \sin x$, $\frac{\delta Q}{\delta x} = -2y \sin x = -2x\sin y$.

Видим, что \textbf{равенство выполняется}.

\paragraph{Пример №4}

\begin{equation}
    \begin{cases}
        x = \phi(t) = 2 \cos t \ x' = -2 \sin t \\
        y = \psi(t) = 2 \sin t \ y' = 2 \cos t \\
        t \in [0; 2\pi]
    \end{cases}
\end{equation}

$\oint y^2 \diff x + 2 x y \diff y = \int\limits_{t_1}^{t_2} [P(x(t), y(t)) x'(t) \diff t + Q(x(t), y(t)) y'(t)] \diff t = \int\limits_{0}^{2 \pi} (4 \sin^2 t (-2 \sin t) + 2 2 \cos t 2 \sin t 2 \cos t) \diff t = 8 \int\limits_{0}^{2 \pi} \sin t (-\sin^2 t + 2 \cos^2 t) \diff t = -8 \int\limits_{0}^{2 \pi} (-1 + 3\cos^2 t) \diff \cos t = 8 \int\limits_{0}^{2 \pi} (1 - 3 \cos^2 t) \diff \cos t = 8 (\cos t - \frac{3 \cos^3 t}{3}) \bigg|_{0}^{2 \pi} = 0$


Мы в самом начале могли проверить:

$\frac{\delta P}{\delta y} = \frac{\delta Q}{\delta x} = 2y$ — следовательно, интеграл по замкнутому контуру равен нулю, и не считать все то, что мы все же посчитали.

\paragraph{Пример №5}

$\oint\limits_{D} x^2 \cos y \diff x + 3y^3 x \diff y = \int\limits_{OA} (\dots) + \int\limits_{AB} (\dots) + \int\limits_{BO} (\dots) = \int\limits_{0}^{1} (x^2 \cos x + 3x^3) \diff x + \int\limits_{1}^{0} (x^2 \cos (2 - x) - 3 (2 - x)^3 x) \diff x + 0 = \dots$. Дальнейшее решение оставляется в качестве упражнения читателю.

Но можно было пойти другим путем: посчитать $\frac{\delta Q}{\delta x} = 2 x \cos y$, $\frac{\delta P}{\delta y} = 9 y^2 x$, так что $\int\int (\frac{\delta Q}{\delta x} - \frac{\delta P}{\delta y}) \diff x \diff y = \int\limits_{0}^{G_1} \diff x \int\limits_{x}^{2 - x} (2x \cos y - 9y^2 x) \diff y = 2 x \sin y - \frac{9 y^3}{3} x \bigg|_{y = x}^{2 - x} = 2 x \sin (2 - x) - 3 (2 - x)^3 x - 2x \sin x - 3x^4$ — далее это нужно интегрировать по $x$.

\section{Практическое занятие — 05.10.2023}

\subsection{Двойные и тройные интегралы}

\subsubsection{№3618}

Сфера $x^2 + y^2 + z^2 = 4 R z - 3R^2$

Конус $z^2 = 4 (x^2 + y^2)$

$x^2 + y^2 + z^2 - 4Rz+4R^2 = 4R^2 - 3R^2$

$x^2 + y^2 + (z - 2R)^2 = R^2$

$\rho^2 \cos^2 \phi + \rho^2 \sin^2 \phi + z^2 = 4 R z - 3R^2$

$\rho^2 + z^2 - 4 R z + 4 R^2 = R^2$

$\rho^2 + (z - 2R)^2 = R^2$

$z = 2R \pm \sqrt{R^2 - \rho^2}$

$V_1 = \int\limits_{0}^{2 \pi} \diff \phi \int\limits_{0}^{R} \rho \diff \rho \int\limits_{2 \rho}^{2 R + \sqrt{R^2 - \rho^2}} \diff z$

$\int\limits_{0}^{R} (2R + \sqrt{R^2 -\rho^2} - 2 \rho) \rho \diff \rho = (\frac{2R \rho^2}{2}) \bigg|{0}^{R} - \frac{2 \rho^3}{3} \bigg|_{0}^{R} - \frac{1 (R^2 - \rho^2)^{3/2}}{3} \bigg|_{0}^{R} = \frac{1}{3} R^3 + \frac{1}{3} R^3 = \frac{2}{3} R^3$

$V_1 = \frac{4 \pi R^3}{3}$

$V_2 = \int\limits_{0}^{2 \pi} \diff \phi \int\limits_{0}^{\frac{3}{5} R} \rho \diff \rho \int\limits_{2 \rho}^{2 R - \sqrt{R^2 - \rho^2}} \diff z$

\begin{enumerate}
    \item $\int\limits_{2 \rho}^{2 R - \sqrt{R^2 - \rho^2}} \diff z = (z) \bigg|_{2 \rho}^{2 R - \sqrt{R^2 - \rho^2}} = (2R - \sqrt{R^2 - \rho^2}) - (2 \rho) = 2R - \sqrt{R^2 - \rho^2} - 2\rho$
    \item $\dots$
    \item $\dots$
\end{enumerate}

Далее вычисляем, вычитаем что надо из чего надо, и все должно быть в порядке.

\subsection{Криволинейные интегралы}

$\int\limits_{A}^{B} P(x, y) \diff x + Q(x, y) \diff y = \int\limits_{t_{A}}^{t_{B}} (P(x(t), y(t)) x'_t + Q(x(t), y(t)) y'_t) \diff t$

\begin{equation}
    \begin{cases}
        x = x(t) \ \diff x = x'_t \diff t \\
        y = y(t) \ \diff y = y'_t \diff t
    \end{cases}
\end{equation}

Если же имеем декартовы координаты, например:

\begin{equation}
    \begin{cases}
        x = x \\
        y = y(x) \ \diff y = y'_x \diff x
    \end{cases}
\end{equation}

Тогда:

$\int\limits_{A}^{B} P(x, y) \diff x + Q(x, y) \diff y = \int\limits_{x_{A}}^{x^{B}} (P(x, y(x)) + Q(x, y(x)) y'_x) \diff x$

\subsubsection{№3811}

$\int\limits_{(0, 0)}^{(1, 1)} x y \diff x + (y - x) \diff y$

По следующим кривым:

\begin{enumerate}
    \item $y = x$
    
    $\int\limits_{(0, 0)}^{(1, 1)} xy \diff x + (y - x) \diff y = \int\limits_{0}^{1} (x^2 + (x - x) 2x) \diff x = \int\limits_{0}^{1} (x^2) \diff x = \frac{x^3}{3} \bigg|_{0}^{1} = \frac{1}{3}$
    \item $y = x^2$
    
    $\int\limits_{(0, 0)}^{(1, 1)} x y \diff x + (y - x) \diff y = \int\limits_{0}^{1} (x * x^2 + (x^2 - x) 2x) \diff x = (\frac{3x^4}{4} - \frac{2x^3}{3}) \bigg|_{0}^{1} = \frac{3}{4} - \frac{2}{3} = \frac{1}{12}$
\end{enumerate}

\subsubsection{№3815}

Вычислить $\int \frac{y^2 \diff x - x^2 \diff y}{x^2 + y^2}$ по $L$:

\begin{equation}
    \begin{cases}
        x = 2 \cos t \ x' = - 2 \sin t \\
        y = 2 \sin t \ y' = 2 \cos t \\
        0 \le t \le \pi
    \end{cases}
\end{equation}

$\int \frac{y^2 \diff x - x^2 \diff y}{x^2 + y^2} = \int\limits_{0}^{\pi} \frac{(4 \sin^2 t (-2 \sin t)) - (4 \cos^2 t (2 \cos t))}{4 \cos^2 t + 4 \sin^2 t} = 2 \int\limits_{0}^{\pi} (1 - \cos^2 t) \diff (\cos t) - (1 - \sin^2 t) \diff (\sin t) = 2 ((\cos t - \frac{\cos^3 t}{3}) - (\sin t - \frac{\sin^3 t}{3})) \bigg|_{0}^{\pi} = 2 (-1 + \frac{1}{3} - 0 + 0 - (1 - \frac{1}{3} - 0 + 0)) = 2 (-\frac{2}{3} - \frac{2}{3}) = 2(-\frac{4}{3}) = -\frac{8}{3}$

\subsubsection{№3812}

Вычислить $\int\limits_{(0; 0)}^{(1,1)} 2xy \diff x + x^2 \diff y$ по $L$:

\begin{enumerate}
    \item 

    $\begin{cases}
        y = x \\
        y' = 1
    \end{cases}$

    $\int\limits_{(0; 0)}^{(1,1)} 2xy \diff x + x^2 \diff y = \int\limits_{0}^{1} (2x^2 + x^2) \diff x = (\frac{2x^3}{3} + \frac{x^3}{3}) \bigg|_{0}^{1} = \frac{2}{3} + \frac{1}{3} = 1$
    
    \item 

    $\begin{cases}
        y = x^2 \\
        y' = 2x
    \end{cases}$

    $\int\limits_{(0; 0)}^{(1,1)} 2xy \diff x + x^2 \diff y = \int\limits_{0}^{1} (2x^3 + 2x^3) \diff x = \int\limits_{0}^{1} (4x^3) \diff x = \frac{4x^4}{4} \bigg|_{0}^{1} = 1$

    \item

    $\begin{cases}
        y = x^3 \\
        y' = 3x^2
    \end{cases}$

    $\int\limits_{(0; 0)}^{(1,1)} 2xy \diff x + x^2 \diff y = \int\limits_{0}^{1} (2x^4 + x^2 * (3x^2)) \diff x = \int\limits_{0}^{1} (2x^4 + 3x^{4}) = (\frac{2x^5}{5} + \frac{3x^5}{5}) \bigg|_{0}^{1} = \frac{2}{5} + \frac{3}{5} = 1$
\end{enumerate}

\subsection{Восстановление функции по ее полному дифференциалу}

$\diff u = P \diff x + Q \diff y$, $P = \frac{\delta u}{\delta x}$, $Q = \frac{\delta u}{\delta y}$

$P = \frac{\delta u}{\delta x}$, $u = \int P \diff x = F(x, y) + C(y)$

$\frac{\delta u}{\delta y} = \frac{\delta F}{\delta y} + \frac{\delta C}{\delta y} = Q$, $C = C(y)$

\paragraph{Пример №1}

$\diff u = x^2 \diff x + y^2 \diff x$

$x^2 = \frac{\delta u}{\delta x}$, $y^2 = \frac{\delta u}{\delta y}$

$u = \int x^2 \diff x = \frac{x^3}{3} + C(y)$

$\frac{\delta u}{\delta y} + \frac{\delta C}{\delta x} = y^2$

$C(y) = \int y^2 \diff y = \frac{y^3}{3} + K$

$u = \frac{x^3}{3} + \frac{y^3}{3} + K$

\subsubsection{№3846}

$\diff u = (4x^3 - 4y^2 x) \diff x - (4x^2 y - 4y^3) \diff y$

$u = \int (4x^3 - 4y^2 x) \diff x = \frac{4x^4}{4} - \frac{4y^2 x^2}{2} = x^4 - 2y^2 x^2 + C(y)$

$\frac{\diff u}{\diff y} = -4yx^2 + \frac{\diff C}{\diff y} = -4x^2 y + 4y^3$

$\frac{\diff C}{\diff y} = 4y^3$

$C(y) = 4y^3 \diff y = \frac{4y^4}{4} = y^4 + K$

\textbf{Ответ:} $u = x^3 - 2y^2 x^2 + y^4 + K$

\subsubsection{№3847}

$\diff u = \frac{(x + 2y) \diff x + y \diff y}{(x + y)^2} = \frac{x + 2y}{(x+y)^2} \diff x + \frac{y}{(x + y)^2} \diff y$

Ненужный шаг, мы зря начали это считать: $\frac{\diff P}{\diff y} = \frac{2(x + y)^2 - 2(x + 2y)(x + y)}{(x + y)^4} = \frac{2(x + y - x - 2y)}{(x+y)^3} = -\frac{2y}{(x + y)^3}$, проверим другую частную производную: $\frac{\diff Q}{\diff x} = \frac{-2y(x + y)}{(x+y)^3} = \frac{-2y}{(x+y)^3}$. Они совпадают, значит мы можем решать.

$u = \int \frac{x + y + y}{(x+y)^2} \diff x = \int \frac{x + y}{(x + y)^2} + \int \frac{y}{(x + y)^2} \diff x = \int \frac{\diff (x + y)}{x + y} + y \int \frac{d(x + y)}{(x + y)^2} = \ln |x + y| - \frac{y}{x + y} + C(y)$

$\frac{\diff u}{\diff y} = \frac{1}{x + y} - \frac{(x + y) - y}{(x + y)^2} + \frac{\diff C}{\diff y} = \frac{1}{x + y} - \frac{x}{(x + y)^2} + \frac{\diff C}{\diff y} = \frac{y}{(x + y)^2}$

$\frac{\diff C}{\diff y} = \frac{x + y}{(x + y)^2} - \frac{1}{x + y} = \frac{1}{x + y} - \frac{1}{x + y} = 0$

$C(y) = C$

\textbf{Ответ}: $u = \ln |x + y| - \frac{y}{x + y} + C$

\subsection{Вычисление криволинейных интегралов от полных дифференциалов}

\paragraph{Пример №1} $\int\limits_{(0; 0)}^{(2; 1)} 2 xy \diff x + x^2 \diff y$

Возможны несколько вариантов:

\begin{enumerate}
    \item Восстановить полный дифференциал, после чего вычислить:
    
    $\int\limits_{A}^{B} \diff u = u(B) - u(A)$
    \item Так как путь интегрирования не имеет значения, можем сами его выбирать. Например:
    
    $AO: \begin{cases}
        y = 0 \\
        \diff y = 0
    \end{cases}, AB: \begin{cases}
        x = 2 \\
        \diff x = 0
    \end{cases}$

    $\int\limits_{(0; 0)}^{(2; 1)} 2 xy \diff x + x^2 \diff y = \int\limits_{0}^{2} 2x * 0 \diff x + \int\limits_{0}^{1} 4 \diff y = 0 + 4y \bigg|_{0}^{1} = 0 + 4 = 4$
\end{enumerate}

\subsubsection{№3840}

$\int\limits_{(3;4)}^{(5;12)} \frac{x \diff x}{x^2 + y^2} + \frac{y \diff y}{x^2 + y^2}$

Проверим:

$\frac{\diff P}{\diff y} = \frac{0 - 2xy}{(x^2 + y^2)^2}$

$\frac{\diff Q}{\diff x} = \frac{0 - 2xy}{(x^2 + y^2)^2}$

Пойдем по второму пути:

$AD: \begin{cases}
    x = 3 \\
    \diff x = 0
\end{cases}$

$DB: \begin{cases}
    y = 12 \\
    \diff y = 0
\end{cases}$

$\int\limits_{4}^{12} (\frac{y}{9 + y^2}) \diff y = (\frac{1}{2} \ln |y^2 + 9|) \bigg|_{4}^{12} = (\frac{1}{2} \ln |144 + 9|) - (\frac{1}{2} \ln |16 + 9|)$

$\int\limits_{3}^{5} (\frac{x}{x^2 + 144}) \diff x = (\frac{1}{2} \ln |x^2 + 144|) \bigg|_{3}^{5} = (\frac{1}{2} \ln |25 + 144|) - (\frac{1}{2} \ln |9 + 144|)$

$\int\limits_{(3;4)}^{(5;12)} \frac{x \diff x}{x^2 + y^2} + \frac{y \diff y}{x^2 + y^2} = \frac{1}{2} (\ln|169| - \ln |25|)$

\section{Практическое занятие — 17.10.2023}

\subsection{Дифференциальные уравнения}

\begin{definition}
    \textbf{Дифференциальным уравнением} называется уравнение, связывающее независимую переменную $x$, искомую функцию $y$ и ее какие-то производные:
    $$
    F(x, y, y', y'', \dots) = 0
    $$
    \textbf{Обыкновенное дифференциальное уравнение}:
    $$
    y' = f(x)
    $$
    \textbf{Дифферециальное уравнение в частных производных}:
    $$
    y' = f(x, t)
    $$
\end{definition}

\begin{definition}
    \textbf{Порядком} в дифференциальном уравнении называется порядок наивысшей производной, входящей в это уравнение.
\end{definition}

\begin{definition}
    \textbf{Решением} (интегралом) дифференциального уравнения называется всякая функция, которая будучи подставлена в уравнение обращает его в верное равенство.
\end{definition}

\paragraph{Пример №1}

Пусть

$$y'' - y = 0$$

Проверим:

\begin{enumerate}
    \item $y = x$

    $y' = 1$, $y'' = 0$

    Таким образом, не является решением.
    \item $y = \sin x$

    $y' = \cos x$, $y'' = -\sin x$

    Таким образом, не является решением
    \item $y = e^{x}$

    $y' = e^{x}$, $y'' = e^{x}$

    Является решением.
\end{enumerate}

\begin{definition}
    \textbf{Общим решением} дифференциального уравнения называется функция $y = \phi(x; C)$ ($C = const$), которая
    
    \begin{enumerate}
        \item удовлетворяет дифференциальному уравнению при любом значении $C$;
        \item каково бы не было начальное условие, всегда можно подобрать значение $C_0$, чтобы оно удовлетворяло указанному начальному условию.
    \end{enumerate}
\end{definition}

\paragraph{Пример №1}

$y = x^2 + C$

$y(0) = 4$

$y = x^2 + 4$ — Решение задачи Коши, удовлетворяет нач. усл.

\paragraph{Пример №2}

Найти уравнение кривой, у которой точка пересечения любой касательной с осью $OX$ $K$ равноудалена от нач. координат $O$ и от точки касания $M(x, y)$.

Условие, которое должно быть выполнено: $|OK| = |OM|$

Уравнение касательной:

$Y - y = y' (X - x)$

В т.к $Y = 0$, получаем:

$-y = y'(X - x) \Leftrightarrow x - \frac{y}{y'} = X$. Таким образом, можем теперь задать коордианты т. K: $(x - \frac{y}{y'}; 0)$

$|OK| = |x - \frac{y}{y'}|$

$|KM| = \sqrt{(x - (x - \frac{y}{y'}))^2 + (y - 0)^2}$

$|x - \frac{y}{y'}| = \sqrt{(\frac{y}{y'})^2 + y^2}$

$x^2 - \frac{2xy}{y'} + (\frac{y}{y'})^2 = (\frac{y}{y'})^2 + y^2$

$-\frac{2xy}{y'} = -y^2 + x^2$

$y' = \frac{2xy}{x^2-y^2}$ — Дифференциальное уравнение (I порядок)

\paragraph{Геометрическая интерпретация решения дифференциального уравнения}

Общее решение (общий интеграл) представляет собой семейство кривых на координатной плоскости, а решение задачи Коши представляет собой кривую, проходящую через заданную точку.

\subsubsection{Теорема о существовании и единственности решения}

\begin{theorem}[О существовании и единственности решения]
    Пусть в дифференциальном уравнении $y' = F(x, y)$ функция $F(x, y)$ и ее частная производная $\frac{\delta f}{\delta y}$ непрерывны на открытом множестве $G$. В этом случае
    \begin{enumerate}
        \item Для всякой точки $(x_0, y_0) \in G$ найдется решение $y = y(x)$ дифференциального уравнения $y' = f(x, y)$, удовлетворяющее условию $y_0 = y(x_0)$ (решение задачи Коши).
        \item Если два решения ($y = y_1(x)$, $y = y_2(x)$) дифференциального уравнения $y' = F(x, y)$ совпадают хотя бы для одного решения $x^{*}$. то эти решения совпадают для всех тех значений переменной $x$, для которой они определены.
    \end{enumerate}
\end{theorem}

\subsubsection{Дифференциальный уравнения (I порядок) с разделенными или разделяющимися переменными}

Дифференциальное уравнение (I порядок) может быть записано:

\begin{enumerate}
    \item $y' = f(x; y)$
    \item $p(x; y) \diff x + q(x, y) \diff y = 0$
\end{enumerate}

\begin{definition}
    \textbf{Уравнением с разделяющимися переменными} называют дифференциальное уравнение, в котором функция $f$ может быть разбита на две такие функции, разделенные знаками умножения или деления, что одна из них зависит только от $x$, а другая зависит только от $x$.

    $$y' = f_1(x) * f_2(y) \ \ \textbf{(1)}$$
    $$p_1(x) * p_2(y) + q_1(x) * q_2(y) \diff y = 0 \ \ \textbf{(2)}$$
\end{definition}

Разделение переменных:

\begin{enumerate}
    \item $\frac{\diff y}{\diff x} = f_1(x) * f_2(y)$

    $\frac{\diff y}{f_2(y)} = f_1(x) \diff x$
    \item $p_1(x) p_2(y) \diff x = -q_1(x) q_2(y) \diff y$

    $\frac{p_1(x) \diff x}{q_1(x)} = -\frac{q_2(y) \diff y}{p_2(y)}$
\end{enumerate}

После этого мы можем интегрировать левую и правую часть:

\begin{enumerate}
    \item $\int \frac{\diff y}{f_2(y)} = \int f_1(x) \diff x$
    \item $\int \frac{p_1(x) \diff x}{q_1(x)} = -\int \frac{q_2(y) \diff y}{p_2(y)}$
\end{enumerate}

Получим:

\begin{enumerate}
    \item $\Phi_2(y) = \Phi_1(x) + C$

    или

    $\Phi_2(y) - \Phi_1(x) = C$
    \item $F_1(x) = F_2(y) + C$
\end{enumerate}

\paragraph{Пример №1}

$\sqrt{y^2 + 1} \diff x = xy \diff y$

Выполним разделение переменных:

$\frac{\diff x}{x} = \frac{y \diff y}{\sqrt{y^2 + 1}}$

Интегрируем полученное:

$\int \frac{\diff x}{x} = \int \frac{y \diff y}{\sqrt{y^2 + 1}}$

$\ln |x| = (y^2 + 1)^{-\frac{1}{2}} (y^2 + 1)$

$\ln |x| = \frac{1}{2} \frac{(y^2 + 1)^{1/2}}{1/2} + C$

$\ln |x| = (y^2 + 1)^{1/2} + C$

\textbf{Ответ}: $\ln Cx = \sqrt{y^2 + 1}$

\paragraph{Замечание о приведении к уравнениям с разделющимися переменными}

Уравнение вида $y' = f(ax + by + c)$ можно привести заменой $z = ax + by$ или $ax + by + C$ к уравнению с разделяющимися переменными.

\paragraph{Пример №2}

$(x + 2y) y' = 1$

$y' = \frac{1}{x + 2y}$

Заменим: $z = x + 2y$

$\frac{z - x}{2} = y$

$\frac{z' - 1}{2} = y'$

$\frac{z' - 1}{2} = \frac{1}{z}$

$\frac{\diff z}{\diff x} = z' = \frac{2}{z} + 1 = \frac{2 + z}{z}$

$\frac{z}{z + 2} \diff z = \diff x$

Переменные теперь разделены и их можно проинтегрировать:

$\int \frac{z + 2 - 2}{z + 2} \diff z = \int \diff x$

$\int (1 - \frac{2}{z + 2}) \diff z = \int \diff x$

$\int \frac{\diff z}{z + 2} = \ln |z + 2|$, $\int \diff z = 2$

$z - 2 \ln |z + 2| = x + C$

$x + 2y - 2 \ln |x + 2y + 2| = x + C$

\textbf{Ответ:} $2y - 2 \ln |x + 2y + 2| = C$

\subsubsection{Однородные дифференциальные уравнения (I порядок)}

\begin{definition}
    \textbf{Функция $f(x y)$ называется однородной функцией} $n$-го измерения относительно переменных $x$ и $y$, если для любой $\lambda$ справедливо следующее:
    $$
    f(\lambda x, \lambda y) = \lambda^{n} f(x; y)
    $$
\end{definition}

Пример однородной функции второго измерения:

$$f(x, y) = \sqrt[3]{x^6 + y^6}$$

$$f(\lambda x, \lambda y) = \sqrt[3]{\lambda^6 x^6 + \lambda^6 y^6} = \sqrt[3]{\lambda^6 (x^6 + y^6)} = \lambda^2 \sqrt{x^6 + y^6}$$

Пример функции, не являющейся однородной:

$$f(x, y) = \sqrt[3]{x^6 + y^3}$$

$$f(\lambda x, \lambda y) = \sqrt[3]{\lambda^6 x^6 + \lambda^3 y^3} = \lambda^3 \sqrt{\lambda^3 x^3 + y^6}$$

\begin{definition}
    \textbf{Дифференциальное уравнение (I порядок) называется однородным} относительно $x$ и $y$, если $f(x, y)$ является однородной функцией нулевого измерения:

    $$f(\lambda x, \lambda y) = \lambda^{0} f(x, y)$$
\end{definition}

Пример однородного дифференциального уравнения:

$$f(x, y) = \frac{x^2 - y^2}{2xy}$$

$$f(\lambda x, \lambda y) = \frac{\lambda^2 x^2 - \lambda^2 y^2}{2 \lambda x \lambda y} = \frac{x^2 - y^2}{2xy}$$

\paragraph{Решение однородных дифференциальных уравнений (I порядок)} 

Необходимо заменить:

$$t = \frac{y}{x}$$

$$y = tx$$

$$y' = t' x + t$$

Получим:

$$t' * x + t = f(t)$$

$$t' * x = f(t) - t$$

$$\frac{\diff t}{\diff x} * x = f(t) - t$$

$$\int \frac{\diff t}{f(t) - t} = \int \frac{\diff x}{x}$$

Переменные разделились и мы интегрируем, не забывая обратно заменить $t$.

\paragraph{Пример №1}

$y' = \frac{x + 2y}{x}$

$y' = 1 + 2 * \frac{y}{x}$

Заменим:

$t = \frac{y}{x}$, $y = tx$, $y' = t' x + t$

Получим:

$t'x + t = 1 + 2t$

$t' * x = 1 + t$

$\frac{\diff t}{\diff x} * x = 1 + t$

Выполним окончательное разделение:

$\frac{\diff t}{t + 1} = \frac{\diff x}{x}$

Теперь мы можем интегрировать:

$\int \frac{\diff t}{t + 1} = \int \frac{\diff x}{x}$

$\ln |t + 1| = \ln |x| + \ln C$

$\ln |t + 1| = \ln Cx$

$t + 1 = Cx$

$\frac{y}{x} = Cx - 1$

\textbf{Ответ:} $y = Cx^2 - x$

\pagebreak
\section{Лекция — 31.10.2023}

\subsection{Дифференциальные уравнения}

\subsubsection{Уравнения, приводимые к однородным}

$y' = f(\frac{a_1 x + b_1 y + c_1}{a_2 x + b_2 y + c_2})$

Возможны варианты в зависимости от значения $\begin{vmatrix} a_1 & b_1 \\ a_2 & b_2 \end{vmatrix}$

\begin{enumerate}
    \item Если этот определитель равен нулю, то это случай, когда $\frac{a_1}{a_2} = \frac{b_1}{b_2}$. В этом случае необходимо делать замену, как будет удобно в каждом конкретном случае.
    
    Допустим, $z = a_1 x + b_1 y + c_1$, выражаем: $y = \frac{1}{b_1} (z - a_1 x - c_1)$, $y' = \frac{z'}{b_1} - \frac{a_1}{b_1}$. 
    
    Если мы все это подставим, то получим \textbf{уравнение в разделяющихся переменных}, которое мы можем решить.

    \paragraph{Пример №1} $y' = \frac{x + 2y - 1}{2x + 4y + 5}$

    Выразим: $z = x + 2y - 1 \implies y = \frac{1}{2} z - \frac{1}{2} x + \frac{1}{2}$, $y' = \frac{1}{2} z' - \frac{1}{2}$. 
    
    И далее подставим: $\frac{1}{2} z' - \frac{1}{2} = \frac{z}{2z + 7} \Longleftrightarrow \frac{1}{2} z' = \frac{z}{2z + 7} + \frac{1}{2} \Longleftrightarrow \frac{1}{2} \frac{\diff z}{\diff x} = \frac{2z + 2z + 7}{2 (2z + 7)}$.

    Теперь можем интегрировать: $\int \frac{2 z + 7}{4 z + 7} \diff z = \int \diff x$. Но дробь неправильная, поэтому выделим целую часть, получим: $\frac{1}{2} \int \frac{(4z + 7) + 7}{4z + 7} \Longleftrightarrow \frac{1}{2} (\int \diff z + \int \frac{7}{4z + 7} \diff z) \Longleftrightarrow \frac{1}{2} (z + \frac{7}{4} \ln |4z + 7|) = x + C$

    Мы получили неявный вид решения дифференциального уравнения. Единственное, что важно — подставить $x + 2y$ вместо $z$ обратно, и всё будет хорошо: $x + 2y - 1 + \frac{7}{4} \ln |4 (x + 2y - 1) + 7| = 2x + K$
    \item Если определитель не равен нулю, то есть коэффициенты не пропорциональны: $\frac{a_1}{a_2} \ne \frac{b_1}{b_2}$, то $x = u + \alpha$, $y = v + \beta$, где $\alpha$ и $\beta$ — решения системы
    
    $$\begin{cases}
        a_1 \alpha + b_1 \beta + c_1 = 0 \\
        a_2 \alpha + b_2 \beta + c_2 = 0
    \end{cases}$$

    После того, как мы всё подставим, у нас должно будет получиться что-то вроде $\frac{k_1 u + k_2 v}{k_3 u + k_4 v}$. То есть, свободных членов не должно остаться. Делим обе части на $u$, получаем: $\frac{k_1 + k_2 \frac{v}{u}}{k_3 + k_4 \frac{v}{u}}$ — однородное уравнение. Пользуясь заменой $\frac{v}{u} = t$, оставшееся бодренько решаем: $v = ut$, $v_u' = 1 * t + ut'_u$

    \paragraph{Пример №2} $(2x - 4y + 6) \diff x + (x + y - 3) \diff y = 0$

    Приведем необходимому виду: $\frac{\diff y}{\diff x} = \frac{2x - 4y + 6}{x + y - 3}$. После чего составим и решим систему:

    $$
    \begin{cases}
        2 \alpha - 4 \beta + 6 = 0 \\
        \alpha + \beta - 3 = 0
    \end{cases} \Longleftrightarrow \begin{cases}
        6 \alpha - 6 = 0 \\
        1 + \beta - 3 = 0
    \end{cases} \Longleftrightarrow \begin{cases}
        \alpha = 1 \\
        \beta = 2
    \end{cases}
    $$

    Выполним замену $x = u + 1$, $y = v + 2$, $\diff x = \diff u$, $\diff y = \diff v$:

    $\frac{\diff v}{\diff u} = \frac{2 (u + 1) - 4 (v + 2) + 6}{u + 1 + v + 2 - 3} \Longleftrightarrow \frac{\diff v}{\diff u} = \frac{2 u - 4v}{u + v}$. У нас, как и ожидалось, пропали константы. Мы сделали всё правильно. Если бы они не пропали — значит, мы что-то сделали не так. У нас теперь есть однородное уравнение, которое мы легко можем решать: $\frac{\diff v}{\diff u} = \frac{2 u - 4v}{u + v}$. Поделим на $u$ всю правую часть, получим: $\frac{\diff v}{\diff u} = \frac{2 - 4 \frac{v}{u}}{1 + \frac{v}{u}}$, $\frac{v}{u} = t$, $v = ut$, $\frac{\diff v}{\diff u} = t + u \frac{\diff t}{\diff u}$. 
    
    Осталось сделать уравнение с разделяющимися переменными для этих $t$ и $u$: $t + u \frac{\diff t}{\diff u} = \frac{2 - 4t}{1 + t} = \frac{2 - 4t - t - t^2}{t + 1} = - \frac{t^2 + 5t - 2}{t + 1}$. Разделяем переменные: $\frac{t + 1}{t^2 + 5t - 2} \diff t = \frac{\diff u}{u}$. Необходимо интегрировать:  $\int \frac{t + 1}{t^2 + 5t - 2} \diff t = \int \frac{\diff u}{u} \Longleftrightarrow \frac{1}{2} \int \frac{2 t + 5}{t^2 + 5t 2 2} \diff t - \frac{3}{2} \int \frac{\diff t}{t^2 + 5t - 2} = \ln + \ln c$. Далее: $\frac{1}{2} \ln |t^2 + 5t - 2| + \frac{1}{2} * \frac{2}{\sqrt{33}} \ln |\frac{t + \frac{5}{2} - \frac{\sqrt{33}}{2}}{t + \frac{5}{2} + \frac{\sqrt{33}}{2}}| = c_n c_u$. Получили ужас. Не будем ничего делать, лишь вернемся к переменной $u$ и потом к $x$, $y$.
\end{enumerate}

\subsubsection{Линейные уравнения и уравнения Бернулли. Метод вариации произвольной постоянной}

\begin{definition}{\textbf{Линейное уравнение}}
    имеет вид:

    $$y' + p(x)y = q(x)$$
\end{definition}

\begin{definition}{\textbf{Уравнение Бернулли}}
    имеет вид:

    $$y' + p(x)y = q(x) y^{\alpha}$$
\end{definition}

Для обеих видов уравнений:

$y = y_{oo} + \tilde{y}$, $y' + p(x) y = 0$, $\frac{\diff y}{\diff x} = -p(x) y$, $\int \frac{\diff y}{y} = - \int p(x) \diff x$. Получилось уравнение с разделяющимися переменными, $y_{oo} = \Phi(x) + C$.

\textbf{Метод вариации произвольной переменной} заключается в том, что решение уравнения будем искать в виде похожем на $y_{oo}$, только константу $C$ мы будем рассматривать как функцию, зависящую от $x$: $y = \Phi(x) + C(x)$, $y' = \Phi'(x) + C'(x)$. Эти значения мы подставляем в то уравнение, с которым работаем. Если у нас в итоге что-то сокращается — это хорошо; значит, что мы на верном пути. У нас получится уравнение относительно $C(x)$ и $x$

\paragraph{Пример №1}

$y' - \frac{y}{x} = x^2 + 2x$

\begin{enumerate}
    \item $y' - \frac{y}{x} = 0$, $\frac{\diff y}{\diff x} = \frac{y}{x}$, $\int \frac{\diff y}{y} = \int \frac{\diff x}{x}$, $\ln |y| = \ln |x| + \ln C$, $\ln y = \ln C_x$, $y_{oo} = Cx$ — общее решение уравнения с нулём в правой части.
    \item $$
    \begin{cases}
        y = C(x) \\
        y' = C'(x) x + C(x)
    \end{cases}$$

    Подставляем в исходное уравнение: $C'(x) + C(x) - \frac{C(x) x}{x} = x^2 + 2x \Longleftrightarrow \frac{\diff C(x)}{\diff x} = x + 2 \Longleftrightarrow \int \diff C(x) = \int (x + 2) \diff x$, $C(x) = \frac{x^2}{2} + 2x + K$. Теперь подставим обратно, получим: $y = (\frac{x^2}{2} + 2x + K) * x = \frac{x^3}{2} + 2x^2 + Kx$, где $y_{oo} = Kx$, $\tilde{y} = \frac{x^2}{2} + 2x^2$
\end{enumerate}

\paragraph{Пример №2}

$x y' - 2y = 2x^4$. Но нам это не нравится, уединим $y'$, чтобы всё было симпатичней: $y' - \frac{2y}{x} = 2x^3$

\begin{enumerate}
    \item Решим уравнение $y' - \frac{2y}{x} = 0$ и запишем его ответ: $\frac{\diff y}{y} = \int \frac{2 \diff x}{x}$, $\ln |y| - 2 \ln x + \ln C$, $\ln y = \ln C x^2$, $y_{oo} = Cx^2$
    \item $$\begin{cases}
        y = C(x) * x^2 \\
        y' = C'(x) x^2 + 2x * C(x)
    \end{cases}$$

    $C'(x) x^2 + 2x C(x) - \frac{2 C(x) x^2}{x} = 2x^3$

    $\frac{\diff C}{\diff x} = 2x$, $\int \diff C = 2 \int x \diff x \Longleftrightarrow C(x) = \frac{2x^2}{2} + K$

    Получаем: $y = (x^2 + K) x^2 = x^4 + Kx^2$, где $\tilde{y} = x^4$, $y_{oo} = Kx^2$
\end{enumerate}

\subsubsection{Метод Бернулли}

$y = u * v$, $y' = u' v + u v'$

Имеем уравнение: $y' + p(x) y = q(x) y^{\alpha}$. Подставим, получим: $u' v + u v' + p(x) u v = q(x) u^\alpha v^\alpha$. Далее выберем такие функции $u$ и $v$, чтобы $uv' + p(x) u v$ занулилось.

\begin{enumerate}
    \item $uv' = -p(x) u v$

    $\frac{\diff v}{\diff x} = - p(x) v$

    $\int\frac{\diff v}{v} = - \int p(x) \diff x$

    Результат должен быть записан: $v = \Phi(x)$ без константы.
    \item После наших преобразований имеем: $u' v = q(x) u^\alpha v^\alpha \Longrightarrow u' = q(x) u^\alpha v^{\alpha - 1}$
    
    $\frac{\diff u}{\diff x} = q(x) u^\alpha \Phi^{\alpha - 1}(x)$

    $\int \frac{\diff u}{u^{\alpha}} = \int q(x) \Phi^{\alpha - 1}(x) \diff x$

    У нас получится решение: $u = F(x) + K$. Таким образом, итоговый ответ: $y = (F(x + K)) \Phi(x)$
\end{enumerate}

\paragraph{Пример №1} $y' - \frac{y}{x} = x^2 + 2x$

$y = u v$, $y' = u' v + u v'$

$u'v + uv' - \frac{uv}{x} = x^2 + 2x$

\begin{enumerate}
    \item Обнуляем $uv' - \frac{uv}{x}$: $\frac{\diff v}{\diff x} = \frac{v}{x} \Longleftrightarrow \int \frac{\diff v}{v} = \int \frac{\diff x}{x} \Longleftrightarrow v = x$
    \item Переписываем наше уравнение: $u' v = x^2 + 2x$

    Подставляем: $\frac{\diff u}{\diff x} * x = x^2 + 2x$

    $\diff u = (x + 2) \diff x$

    $u = \frac{x^2}{2} + 2x + K$

    $y = u v = (\frac{x^2}{2} + 2x + K) x = \frac{x^3}{2} + 2x^2 + K x$
\end{enumerate}

\paragraph{Пример №2} Если имеем уравненине в правой части вроде $y' + 2y = y^2 e^{x}$, то мы делаем всё то же самое: $y = uv$, $y' = uv' + uv'$

$u' v + uv' + 2 u v = u^2 v^2 e^x$

\begin{enumerate}
    \item $\frac{\diff v}{v} = -2 \diff x$

    $\ln v = - 2 x$

    $v = e^{- 2x}$
    \item $u' v = u^2 v^2 e^{x} \Longleftrightarrow u' = u^2 v e^{x}$

    $\frac{\diff u}{\diff x} = u^2 e^{-2x} * e^{x}$

    $\int \frac{\diff u}{u^2} = \int e^{-x} \diff x$

    $\frac{1}{u} = e^{-x} + C \Longleftrightarrow \frac{1}{u} = C + \frac{1}{e^{x}} = \frac{C e^{x} + 1}{e^{x}} \Longleftrightarrow u = \frac{e^{x}}{C e^{x} + 1}$

    $y = u v = \frac{e^{x}}{C e^{x} + 1} e^{-2x} = \frac{1}{e^{x} (C e^{x} + 1)}$
\end{enumerate}

\paragraph{Важное замечание}

Иногда может оказаться, что уравнение не является линейным, если мы считаем $y$ за функцию, а $x$ за переменную, но можно принять $x$ за функцию и $y$ за переменную и сделать его линейным.

\end{document}