\documentclass{article}

\usepackage[T2A]{fontenc}
\usepackage[utf8]{inputenc}
\usepackage[russian]{babel}

\usepackage{tabularx}
\usepackage{amsmath}
\usepackage{pgfplots}
\usepackage{geometry}
\usepackage{multicol}
\geometry{
    left=1cm,right=1cm,top=2cm,bottom=2cm
}
\newcommand*\diff{\mathop{}\!\mathrm{d}}

\newtheorem{definition}{Определение}
\newtheorem{theorem}{Теорема}

\DeclareMathOperator{\sign}{sign}

\usepackage{hyperref}
\hypersetup{
    colorlinks, citecolor=black, filecolor=black, linkcolor=black, urlcolor=black
}

\title{Интегралы и дифференциальные уравнения}
\author{Лисид Лаконский}
\date{October 2023}

\begin{document}
\raggedright

\maketitle

\tableofcontents
\pagebreak

\section{Лекция — 03.10.2023}

\subsection{Криволинейные интегралы (II род)}

На прошлой паре мы записывали:

$\int\limits_{M}^{N} P(x y) \diff x + Q(x, y) \diff y$ (по пути $L$) $= \int\limits_{t_1}^{t_2} P(x(t), y(t)) x'_t + Q(x(t), y(t) y'_t) \diff t$, $\begin{cases}
    x = x(t) \ \diff x = x' \diff t \\
    y = y(t) \ \diff y = y' \diff t \\
    t_1 \le t \le t_2
\end{cases}$

Но можно иначе:

$\begin{cases}
    x = x \ \diff x = \diff x \\
    y = y(x) \ \diff y = y' \diff x \\
    x_1 \le x \le x_2
\end{cases}, \int\limits_{M(x_1, y_1)}^{N(x_2, y_2)} P(x, y) \diff x + Q(x, y) \diff y = \int\limits_{x_1}^{x_2} (P(x, y(x)) + Q(x, y(x)) y'_x) \diff x$ по $L: y = y(x)$

\subsubsection{Условие независимости криволинейных интегралов от пути интегрирования}

\textbf{Если под знаком интеграла находится полный дифференциал какой-то функции, то ответ не будет зависеть от пути интегрирования, а только от начальной и конечной точки}.

$\int\limits_{(x_1, y_1)}^{(x_2, y_2)} \diff u = u \bigg|_{(x_1, y_1)}^{(x_2, y_2)} = u(x_2, y_2) - u(x_1. y_1)$

\textbf{Во всех других случаях криволинейный интеграл зависит от пути интегрирования.}

\textbf{Полный дифференциал} функции двух переменных:

$\diff u = \frac{\delta u}{\delta x} \diff x + \frac{\delta u}{\delta y} \diff y$

При этом:

$\frac{\delta}{\delta y} (\frac{\delta u}{\delta x}) = \frac{\delta}{\delta x} (\frac{\delta u}{\delta y})$

\begin{theorem}
    Если во всех точках области $G$ $P(x, y)$, $Q(x, y)$ частные производные $\frac{\delta P}{\delta y}$, $\frac{\delta Q}{\delta x}$ являются непрерывны, то необходимым и достаточным условием того, что криволинейный интеграл не зависит от пути интегрирования является выполнение условия $\frac{\delta P}{\delta y} = \frac{\delta Q}{\delta x}$ во всех точках области $G$.
\end{theorem}

\begin{theorem}
    При выполнении $\frac{\delta P}{\delta y} = \frac{\delta Q}{\delta x}$ криволиненый интеграл по замкнутому контуру равен нулю: $\oint (P \diff x + Q \diff y) = 0$
\end{theorem}

\subsubsection{Формула Грина–Остроградского}

$\oint\limits_{C} P(x, y) \diff x + Q(x, y) \diff y = \int\int\limits_{D} (\frac{\delta Q}{\delta x} - \frac{\delta P}{\delta y}) \diff x \diff y = \int\limits_{a}^{b} \diff x \int\limits_{y_1(x)}^{y_2(x)} \frac{\delta P}{\delta y} \diff y = \int\limits_{a}^{b} (P(x, y_2(x)) - P(x, y_1(x))) \diff x$, где $D$ — замкнутая область, ограниченная контуром $C$.

\textbf{Примечание:} функции $P(x, y)$, $Q(x, y)$ должны быть определены и непрерывны в области $D$ и, кроме того, иметь в ней непрерывные частные производные $\frac{\delta P}{\delta y}$, $\frac{\delta Q}{\delta x}$.

\paragraph{Следствие из данной формулы} Если $P = 0$, $Q = x$, то $\oint x \diff y = \int\int \diff x \diff y$ — площадь, ограниченная областью.

\subsubsection{Примеры}

\paragraph{Пример №1}

$\int\limits_{(1, 1)}^{(2, 2)} \frac{y \diff x + x \diff y}{x^2 + y^2}$ (по прямой $y = x$) $= \int \frac{y \diff x}{x^2 + y^2} + \frac{x \diff y}{x^2 + y^2} = \int\limits_{1}^{2} (\frac{x}{x^2 + x^2} + \frac{x}{x^2 + x^2}) \diff x = \int\limits_{1}^{2} \frac{2x \diff x}{2x^2} = \ln |x| \bigg|_{1}^{2} = \ln 2$

\paragraph{Пример №2}

$\int\limits_{(0, 0)}^{(1, 1)} 2x^2 y \diff x + x \sqrt{y} \diff y$. Вычислить данный интеграл по:

\begin{enumerate}
    \item прямой $y = x$;
    
    $\int\limits_{0}^{1} (2x^2 x + x \sqrt{x}) \diff x = \int\limits_{0}^{1} (2x^3 + x^{3/2}) \diff x = \frac{2x^4}{4/2} + \frac{2x^{5/2}}{5} \bigg|_{0}^{1} = \frac{9}{10}$
    \item параболе $y = x^2$;
    
    $\int\limits_{0}^{1} (2x^2 x^2 + x \sqrt{x^2} 2x) \diff x = \int\limits_{0}^{1} (2x^4 + 2x^3) \diff x = (\frac{2x^{5}}{5} + \frac{2x^{4}}{4}) \bigg|_{0}^{1} = \frac{9}{10}$
    \item ломанной линии, образованной $y = x$, $y = x^2$
    
    $\int\limits_{0}^{B} + \int\limits_{B}^{A} = \int\limits_{0}^{1} \sqrt{y} \diff y = \frac{2y^{3/2}}{3} \bigg|_{0}^{1} = \frac{2}{3}$
\end{enumerate}

\paragraph{Пример №3}

Пусть $P = (2x \cos y - y^2 \sin x)$, $Q = (2y \cos x - x^2 \sin y)$. Проверить независимость от пути интегрирования:

$\frac{\delta P}{\delta y} = -2x \sin y - 2 y \sin x$, $\frac{\delta Q}{\delta x} = -2y \sin x = -2x\sin y$.

Видим, что \textbf{равенство выполняется}.

\paragraph{Пример №4}

\begin{equation}
    \begin{cases}
        x = \phi(t) = 2 \cos t \ x' = -2 \sin t \\
        y = \psi(t) = 2 \sin t \ y' = 2 \cos t \\
        t \in [0; 2\pi]
    \end{cases}
\end{equation}

$\oint y^2 \diff x + 2 x y \diff y = \int\limits_{t_1}^{t_2} [P(x(t), y(t)) x'(t) \diff t + Q(x(t), y(t)) y'(t)] \diff t = \int\limits_{0}^{2 \pi} (4 \sin^2 t (-2 \sin t) + 2 2 \cos t 2 \sin t 2 \cos t) \diff t = 8 \int\limits_{0}^{2 \pi} \sin t (-\sin^2 t + 2 \cos^2 t) \diff t = -8 \int\limits_{0}^{2 \pi} (-1 + 3\cos^2 t) \diff \cos t = 8 \int\limits_{0}^{2 \pi} (1 - 3 \cos^2 t) \diff \cos t = 8 (\cos t - \frac{3 \cos^3 t}{3}) \bigg|_{0}^{2 \pi} = 0$


Мы в самом начале могли проверить:

$\frac{\delta P}{\delta y} = \frac{\delta Q}{\delta x} = 2y$ — следовательно, интеграл по замкнутому контуру равен нулю, и не считать все то, что мы все же посчитали.

\paragraph{Пример №5}

$\oint\limits_{D} x^2 \cos y \diff x + 3y^3 x \diff y = \int\limits_{OA} (\dots) + \int\limits_{AB} (\dots) + \int\limits_{BO} (\dots) = \int\limits_{0}^{1} (x^2 \cos x + 3x^3) \diff x + \int\limits_{1}^{0} (x^2 \cos (2 - x) - 3 (2 - x)^3 x) \diff x + 0 = \dots$. Дальнейшее решение оставляется в качестве упражнения читателю.

Но можно было пойти другим путем: посчитать $\frac{\delta Q}{\delta x} = 2 x \cos y$, $\frac{\delta P}{\delta y} = 9 y^2 x$, так что $\int\int (\frac{\delta Q}{\delta x} - \frac{\delta P}{\delta y}) \diff x \diff y = \int\limits_{0}^{G_1} \diff x \int\limits_{x}^{2 - x} (2x \cos y - 9y^2 x) \diff y = 2 x \sin y - \frac{9 y^3}{3} x \bigg|_{y = x}^{2 - x} = 2 x \sin (2 - x) - 3 (2 - x)^3 x - 2x \sin x - 3x^4$ — далее это нужно интегрировать по $x$.

\end{document}