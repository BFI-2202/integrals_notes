\documentclass{article}

\usepackage[T2A]{fontenc}
\usepackage[utf8]{inputenc}
\usepackage[russian]{babel}

\usepackage{tabularx}
\usepackage{amsmath}
\usepackage{pgfplots}
\usepackage{geometry}
\usepackage{multicol}
\geometry{
    left=1cm,right=1cm,top=2cm,bottom=2cm
}
\newcommand*\diff{\mathop{}\!\mathrm{d}}

\newtheorem{definition}{Определение}
\newtheorem{theorem}{Теорема}

\DeclareMathOperator{\sign}{sign}

\usepackage{hyperref}
\hypersetup{
    colorlinks, citecolor=black, filecolor=black, linkcolor=black, urlcolor=black
}

\title{Интегралы и дифференциальные уравнения}
\author{Лисид Лаконский}
\date{November 2023}

\begin{document}
\raggedright

\maketitle

\tableofcontents
\pagebreak

\section{Практическое занятие — 02.11.2023}

\subsection{Дифференциальные уравнения}

\subsubsection{С разделяющимися переменными}

\begin{enumerate}
    \item $y' = f(x, y)$

    $\frac{\diff y}{\diff x} = f_1(x) * f_2(y)$
    
    $\frac{\diff y}{f_2(y)} = f_1(x) \diff x$

    $\int \frac{\diff y}{f_2(y)} = \int f_1(x) \diff x$

    $\Phi_2(y) = \Phi_1(x) + C$
    \item $p(x, y) \diff x + q (x, y) \diff y = 0$

    $p_1(x) p_2(y) \diff x + q_1 (x) q_2 (y) \diff y = 0$

    $\int \frac{p_1(x)}{q_1(x)} \diff x = - \int \frac{q_2(y)}{p_2(y)} \diff y$

    $F_1(x) = F_2(y) + C$
\end{enumerate}

\begin{minipage}{0.49\textwidth}
    \paragraph{Пример №1} $y' = \frac{x}{y}$

    $\frac{\diff y}{\diff x} = \frac{x}{y}$
    
    $y \diff y = x \diff x$
    
    $\int y \diff y = \int x \diff x$
    
    $\frac{y^2}{2} = \frac{x^2}{2} + \frac{C}{2}$
    
    $y^2 = x^2 + C$
    
    \paragraph{Пример №2} $\frac{\diff y}{\diff x} = \frac{y}{x}$
    
    $\int \frac{\diff y}{y} = \int \frac{\diff x}{x}$
    
    $\ln |y| = \ln |x| + C$
    
    $y = C x$
    
    \paragraph{Пример №3} $(1 + y^2) x \diff x + (1 + x^2) \diff y = 0$
    
    $\int \frac{x \diff x}{1 + x^2} = - \int \frac{\diff y}{1 + y^2}$
    
    $\frac{1}{2} \ln |x^2 + 1| = C - \arctg y$
    
    \paragraph{Пример №4} $y' = (4x + y + 1)^2$, $y' = f(ax + by + c)$
    
    $z = 4x + y + 1$
    
    $y = z - 4x - 1$, $y' = z' - 4$
    
    $z' - 4 = z^2$, $\frac{\diff z}{\diff x} = z^2 + 4$, $\int \frac{\diff z}{z^2 + 2^2} = \int \diff x$
    
    $\frac{1}{2} \arctg \frac{z}{2} = x + C$
    
    $\frac{1}{2} \arctg \frac{4x + y + 1}{2} = x + C$

    \paragraph{Пример №3910}
    $y' + \sin \frac{x + y}{2} = \sin \frac{x - y}{2}$

    $y' = \sin \frac{x - y}{2} - \sin \frac{x + y}{2}$

    $y' = 2 \sin \frac{x - y - (x + y)}{2 * 2} \cos \frac{x - y + x + y}{2 * 2}$

    $y' = 2 \sin (\frac{-y}{4}) \cos (\frac{x}{2})$

    $y' = -2 \sin \frac{y}{4} \cos \frac{x}{2}$

    $\frac{\diff y}{\diff x} = - 2 \sin \frac{y}{4} \cos \frac{x}{2}$

    $\diff y = -2 \sin \frac{y}{4} \cos \frac{x}{2} \diff x$

    $\frac{\diff y}{2 \sin \frac{y}{2}} = -\cos \frac{x}{2} \diff x$

    $\frac{\diff \frac{1}{2} y}{\sin \frac{y}{2}} = -\cos \frac{x}{2} \diff x$

    $t = \frac{1}{2} y, t' = \frac{1}{2} \frac{y^2}{2} = \frac{y^2}{4} \diff t$

    $\int \frac{\diff t}{\sin t} = - 2 \int \cos \frac{x}{2} \diff \frac{x}{2}$

    $\ln \tg \frac{t}{2} = - 2 \sin \frac{x}{2} + C$

    $\ln \tg \frac{y}{4} = - 2 \sin \frac{x}{2} + C$
\end{minipage}
\begin{minipage}{0.49\textwidth}
    \paragraph{Пример №3901} $(xy^2 + x) \diff x + (y - x^2 y) \diff y = 0$

    $(y - x^2 y) \diff y = (-xy^2 - x) \diff x$

    $(1 - x)y \diff y = x (- y^2 - 1) \diff x$

    $\int \frac{y \diff y}{y^2 + 1} = - \int \frac{x \diff x}{x^2 - 1}$

    $\ln |y^2 + 1| = \ln |x^2 - 1| + C$
    \paragraph{Пример №3902}
    $xyy' = 1 - x^2$

    $\frac{x y \diff y}{\diff x} = 1 - x^2$

    $x y \diff y = (1 - x^2) \diff x$

    $y \diff y = \frac{1 - x^2}{x} \diff x$

    $\int y \diff y = \int \frac{1 - x^2}{x} \diff x$

    $\frac{y^2}{2} = \int (\frac{1}{x} - x) \diff x$

    $\frac{y^2}{2} = \ln |x| - \frac{x^2}{2} + C$

    $y^2 = 2 \ln x - x^2 + C$
    \paragraph{Пример №3903}
    $yy' = \frac{1 - 2x}{y}$

    $\frac{y \diff y}{\diff x} = \frac{1 - 2x}{y}$

    $y \diff y = \frac{1 - 2x}{y} \diff x$

    $y^2 \diff y = (1 - 2x) \diff x$

    $\int y^2 \diff y = \int (1 - 2x) \diff x$

    $\frac{y^3}{3} = x - \frac{2x^2}{2} + C$
    \paragraph{Пример №3907}
    $\sqrt{1 - y^2} \diff x + y \sqrt{1 - x^2} \diff y = 0$

    $\sqrt{1 - y^2} \diff x = -y\sqrt{1 - x^2} \diff y$

    $\frac{\sqrt{1 - y^2}}{\diff y} = -\frac{y\sqrt{1 - x^2}}{\diff x}$

    $\frac{\sqrt{1 - y^2}}{y \diff y} = -\frac{\sqrt{1 - x^2}}{\diff x}$

    $\frac{y \diff y}{\sqrt{1 - y^2}} = -\frac{\diff x}{\sqrt{1 - x^2}}$

    Пусть $t = 1 - y^2$, $t' = - 2y \diff y$

    $-\frac{1}{2} \int \frac{\diff t}{\sqrt{t}} = -\arcsin x + C$

    $\frac{1}{2} \frac{\sqrt{t}}{1/2} = \arcsin x + C$

    $\sqrt{1 - y^2} = \arcsin x + C$
\end{minipage}

\begin{minipage}{0.49\textwidth}
    \paragraph{Пример №3913}
    $y' \sin x = y \ln y$; $y \bigg|_{x = \frac{\pi}{2}} = e$

    $\frac{\diff y}{\diff x} \sin x = y \ln y$

    $\frac{\sin x}{\diff x} = \frac{y \ln y}{\diff y}$

    $\frac{\diff x}{\sin x} = \frac{\diff y}{y \ln y}$
    
    $\int \frac{\diff x}{\sin x} = \int \frac{\diff y}{y \ln y}$

    $\int \frac{\diff(\ln y)}{\ln y} = \int \frac{\diff x}{\sin x}$

    $\ln |\ln |y|| = \ln | \tg \frac{x}{2}| + \ln C$
    
    $\ln |y| = c * \tg \frac{x}{2}$

    $1 = c * 1 \implies c = 1$

    $\ln y =  \tg \frac{x}{2}$

    \paragraph{Пример №3915} $\sin y \cos x \diff y = \cos y \sin x \diff x$; $y \bigg|_{x = 0} = \frac{\pi}{4}$

    $\frac{\sin y}{\cos y} \cos x \diff y = \sin x \diff x$

    $\frac{\sin y}{\cos y} \diff y = \frac{\sin x}{\cos x} \diff x$

    $\int \frac{\sin y}{\cos y} \diff y = \int \frac{\sin x}{\cos x} \diff x$

    $- \int \frac{\diff (\cos y)}{\cos y} = - \int \frac{\diff (\cos x)}{\cos x}$

    $\ln |\cos y| = \ln |\cos x| + \ln C$

    $\cos y = c \cos x$

    $\cos \frac{\pi}{4} = c \cos 0$

    $\cos \frac{\pi}{4} = c$

    $c = \frac{\sqrt{2}}{2}$

    $\cos y = \frac{\sqrt{2} \cos x}{2}$
\end{minipage}
\begin{minipage}{0.49\textwidth}
    \paragraph{Пример №3932} $y' = 3x - 2y + 5$

    $z = 3x - 2y + 5$

    $y = \frac{3x}{2} + \frac{5}{2} - \frac{z}{2}$

    $y' = \frac{1}{2} (3 - z')$

    $\frac{1}{2} (3 - z') = z$

    $3 - z' = 2z$

    $\frac{\diff z}{\diff x} = 3 - 2z$

    $\frac{\diff z}{3 - 2z} = \diff x$

    $\int \frac{\diff z}{3 - 2z} = \int \diff x$

    $x = -\frac{1}{2} \ln |3 - 2z| + C$

    $x = -\frac{1}{2} \ln |3 - 6x + 4y - 10| + C$
\end{minipage}

\subsubsection{С однородными переменными}

$f(x, y)$, $f(\lambda x, \lambda y) = \lambda^{m} f(x, y)$

$f(\lambda x, \lambda y) = f(x, y)$

$y' = f(\frac{y}{x})$

$t = \frac{y}{x}$

$y = t * x$

$y'_x = t' * x + t * 1$

\begin{minipage}{0.49\textwidth}
\paragraph{Пример №1} $y' = \frac{x}{y} + \frac{y}{x}$

$t' * x + t = \frac{1}{t} + t$

$\frac{\diff t}{\diff x} x = \frac{1}{t}$

$\int t \diff t = \int \frac{\diff x}{x}$

$\frac{t^2}{2} = \ln |x| + \ln |C|$

$t^2 = \ln x^2 + \ln K$

$\frac{y^2}{x62} = \ln K x^2$

$y^2 = x^2 \ln K x^2$
\paragraph{Пример №2} $\frac{\diff x}{\diff y} = \frac{x}{y} + \frac{x^2}{y^2}$

$\frac{x}{y} = t$

$x = yt$

$x'_y = t + y \frac{\diff t}{\diff y}$
\end{minipage}
\begin{minipage}{0.49\textwidth}
    \paragraph{Пример №3934}
    $y' = \frac{y^2}{x^2} - 2$
    
    $t = \frac{y}{x}$, $t^2 = \frac{y^2}{x^2}$
    
    $y' = t^2 - 2$
    
    $t' * x + t = t^2 - 2$
    
    $\frac{\diff t}{\diff x} * x = t^2 - t - 2$
    
    $\frac{\diff t}{t^2 - t - 2} = \frac{\diff x}{x}$
    
    $\int \frac{\diff t}{t^2 - t - 2} = \int \frac{\diff x}{x}$
    
    $t^2 - t - 2 = 0$
    
    $t_1 = \frac{1 + 3}{2} = 2, t_2 = -1$
    
    $\int \frac{\diff t}{(t - 2)(t + 1)} = \int \frac{\diff x}{x}$

    $\frac{1}{(t - 2)(t + 1)} = \frac{a}{t - 2} + \frac{b}{t + 1}$

    $\frac{1}{(t - 2)(t + 1)} = \frac{a t + a + bt - 2b}{(t - 2)(t + 1)}$

    $a = \frac{1}{3}, b = -\frac{1}{3}$

    $\int \frac{\frac{1}{3} \diff t}{t - 2} - \int \frac{\frac{1}{3}}{t + 1} = \int \frac{\diff x}{x}$

    $\ln \sqrt[3]{\frac{t - 2}{t + 1}} = \ln C x$
\end{minipage}

\begin{minipage}{0.49\textwidth}
    \paragraph{Пример №3935}
    $y' = \frac{x + y}{x - y}$

    $y' = \frac{1 + \frac{y}{x}}{1 - \frac{y}{x}}$

    $t' x + t = \frac{1 + t}{1 - t}$

    $\frac{x \diff t}{\diff x} = \frac{1 + t}{1 - t} - t$

    $\frac{x \diff t}{\diff x} = \frac{1 + t^2}{1 - t}$

    $\frac{x}{\diff x} = \frac{1 + t^2}{(1 - t) \diff t}$

    $\frac{\diff x}{x} = \frac{(1 - t) \diff t}{1 + t^2}$

    $\int \frac{(1 - t) \diff t}{1 + t^2} = \int \frac{1 \diff t}{1 + t^2} - \int \frac{t \diff t}{1 + t^2} = \arctan t - \frac{\ln |1 + t^2|}{2} = \arctan \frac{y}{x} - \frac{\ln |1 + \frac{y^2}{x^2}|}{2}$

    $\arctan \frac{y}{x} - \frac{\ln |1 + \frac{y^2}{x^2}|}{2} = \ln x + \ln C$
\end{minipage}
\begin{minipage}{0.49\textwidth}
    \paragraph{Пример №3941}
    $y' = e^{\frac{y}{x}} + \frac{y}{x}$

    $t' x = e^{t}$

    $\frac{x \diff t}{\diff x} = e^{t}$

    $\int \frac{\diff x}{x} = \int \frac{\diff t}{e^{t}}$

    $- \frac{1}{e^{\frac{y}{x}}} = \ln |x| + C$
    \paragraph{Пример №3942}
    $x y' = y \ln \frac{y}{x}$

    $y' = \frac{y}{x} \ln \frac{y}{x}$

    $t' x + t = t \ln t$

    $t' x = t \ln t - t$

    $\frac{x \diff t}{\diff x} = t \ln t - t$

    $\frac{\diff x}{x} = \frac{\diff t}{t(\ln t - 1)}$

    $\int \frac{d (\ln t - 1)}{\ln t - 1} = \int \frac{\diff x}{x}$

    $\ln \ln (t - 1) = \ln C x$

    $\ln t - 1 = Cx$

    $\ln \frac{y}{x} = Cx + 1$
\end{minipage}

\subsubsection{Приводимые к однородным}

$y' = f (\frac{a_1 x + b_1 y + c_1}{a_2 x + b_2 y + c_2})$

\begin{enumerate}
    \item Если $\frac{a_1}{a_2} = \frac{b_1}{b_2}$, то вводим $z$, например, $z = a_1 x + b_1 y + c_1$
    \item Если $\frac{a_1}{a_2} \ne \frac{b_1}{b_2}$, то
    
    $x = u + \alpha$, $y = v + \beta$

    $\begin{cases}
        a_1 \alpha + b_1 \beta + c_1 = 0 \\
        a_2 \alpha + b_2 \beta + c_2 = 0
    \end{cases}$

    Потом подставляем одно в одно место, другое в другое. В результате не должно остаться свободных членов.
\end{enumerate}

\begin{minipage}{0.49\textwidth}
\paragraph{Пример №1} $y' = \frac{2 x + y - 1}{6x + 3y + 2}$

$z = 2x + y - 1$

$3z = 6x + 3y - 3$

$6x + 3y + 2 = 3z + 5$

$y = z - 2x + 1$, $y' = z' - 2$

$z' - 2 = \frac{z}{3z + 5}$

$\frac{\diff z}{\diff x} = \frac{z}{3z + 5} + 2$

$\frac{\diff z}{\diff x} = \frac{7z + 10}{3z + 5}$

$\int \frac{3z + 5}{7z + 10} \diff z = \int \diff x$

$\frac{3}{7} \int \frac{z + \frac{5}{3}}{z + \frac{10}{7}} \diff z = \frac{3}{7} \int \frac{(z + \frac{10}{7}) + \frac{5}{21}}{(z + \frac{10}{7})} \diff z = \frac{3}{7} [\int \diff z + \frac{5}{21} \int \frac{\diff (z + \frac{10}{7})}{\frac{10}{7}}]$

$\frac{3}{7} [z + \frac{5}{21} \ln |z + \frac{10}{7}|] = x + C$
\end{minipage}
\begin{minipage}{0.49\textwidth}
    \paragraph{Пример №2} $(x + y - 1)^2 \diff y = 2 (y + 2)^2 \diff x$

    $\frac{\diff y}{\diff x} = 2 (\frac{y + 2}{x + y - 1})^2$

    $x = u + \alpha$, $y = v + \beta$

    $\begin{cases}
        \beta + 2 = 0 \\
        \alpha + \beta - 1 = 0
    \end{cases}$, $\beta = -2$, $\alpha = 3$

    $x = u + 3$, $y = v - 2$, $\diff x = \diff u$, $\diff y = \diff v$

    $\frac{\diff v}{\diff u} = 2 (\frac{v - 2 + 2}{u + 3 + v - 2 - 1})^2$, $\frac{\diff v}{\diff u} = 2 (\frac{v}{u + v})^2$

    $\frac{\diff u}{\diff v} = \frac{1}{2} (\frac{u + v}{v})^2$, $\frac{\diff u}{\diff v} = \frac{1}{2} (\frac{u}{v} + 1)^2$

    $t = \frac{u}{v}$, $u = v * t$, $\frac{\diff u}{\diff v} = t + v \frac{\diff t}{\diff v}$, $t + v \frac{\diff t}{\diff v} = \frac{1}{2} (t + 1)^2 = \frac{t^2}{2} + t + \frac{1}{2}$, $\frac{t^2 + 1}{2}$, $\int \frac{\diff t}{t^2 + 1} = \frac{1}{2} \int \frac{\diff v}{v}$

    $\arctg t = \frac{1}{2} \ln |v| + \ln C = \ln c \sqrt{v} \Longleftrightarrow \arctg \frac{u}{v} = \ln C \sqrt{v} \Longleftrightarrow \arctg \frac{x - 3}{y + 2} = \ln C \sqrt{y + 2}$
\end{minipage}

\pagebreak
\section{Лекция — 14.11.2023}

\subsection{Линейное уравнение и уравнение Бернулли}

Вид линейного уравнения:

$$y' + p(x) y = q(x)$$

Вид уравнения Бернулли:

$$y' + p(x) y = q(x) y^{\alpha}$$

\paragraph{Примеры}

Допустим, имеем

$$\frac{\diff y}{\diff x} = \frac{1}{x \cos y + \sin 2 y}$$

$$\frac{\diff x}{\diff y} = x \cos y + \sin 2 y$$

$$\frac{\diff x}{\diff y} + (\cos y) x = \sin 2y$$

Получившееся уравнение мы можем решать уже знакомыми способами.

$$x = u v, \ x' = u' v + u v'$$

$$u'v + uv' + u v \cos y = \sin 2y$$

\begin{enumerate}
    \item 

    $\frac{\diff v}{\diff y} = -v \cos y$

    $\int \frac{\diff v}{v} = -\cos y \diff y$

    $\ln v = - \sin y$

    $v = e^{-\sin y}$

    \item
    
    $\frac{\diff u}{\diff y} e^{- \sin y} = \sin 2y$

    $\int \diff u = \int e^{\sin y} \sin 2y \diff y$

    $u = 2 \int e^{\sin y} \cos y \diff y = 2 \int e^{\sin y} \sin y \diff (\sin y) = 2 \int t e^{t} \diff t = \begin{vmatrix}
        u = t & \diff v = e^{t} \diff t
        \diff t = \diff u & v = e^{t}
    \end{vmatrix} = 2 (te^{t} - e^{t}) + C$

    $x = uv = 2 (\sin y e^{\sin y} - e^{\sin y} + C) e^{\sin y} = 2 \sin y - 2 + C e^{- \sin y}$
\end{enumerate}

\subsection{Уравнения в полных дифференциалах}

Пусть $u = u(x, y)$, ее полный дифференциал: $\diff u = \frac{\delta u}{\delta x} \diff x + \frac{\delta u}{\delta y} \diff y$

Если $P(x, y) \diff x + Q(x, y) \diff y = 0$, $\frac{\delta u}{\delta x} \diff x + \frac{\delta u}{\delta y}{\diff y} = 0$, то $\diff u = 0 \implies u = C$

$\frac{\delta}{\delta y} (\frac{\delta u}{\delta x}) = \frac{\delta}{\delta x} (\frac{\delta u}{\delta y})$

Если $\frac{\delta P}{\delta y} = \frac{\delta Q}{\delta x}$, то имеем уравнение в полных дифференциалах

Если мы уверены, что уравнение — уравнение в полных дифференциалах, то мы можем сделать так:

$\diff u = P(x, y) \diff x + Q(x, y) \diff y$

$u = \int\limits_{x_0}^{x} P(x, y_0) \diff x + \int\limits_{y_0}^{y} Q(x, y) \diff y + C = \int\limits_{x_0}^{x} P(x; y) \diff x + \int\limits_{y_0}^{y} Q(x_0; y) \diff y + C$

$x_0$, $y_0$ — координаты некоторой точки, которая находится в области определения функций $P$ и $Q$, частные производные которых в этой точке также непрерывны.

\paragraph{Примеры}

\paragraph{Пример №1}

$$(2xy + y^2) \diff x + (3x^2 + y) \diff y = 0$$

$$\frac{\delta P}{\delta y} = 2x + 2y$$

$$\frac{\delta Q}{\delta x} = 6x$$

Не является уравнением в полных дифференциалах.

\paragraph{Пример №2}

$$(2x + y^2) \diff x + (2xy + y^3) \diff y = 0$$

$$\frac{\delta P}{\delta y} = 2y$$

$$\frac{\delta Q}{\delta x} = 2y$$

Является уравнением в полных дифференциалах.

\begin{enumerate}
    \item
    
    $\frac{\delta u}{\delta x} = 2x + y^2$

    $u = \int (2x + y^2) \diff x = x^2 + y^2 x + C(y)$
    \item 

    $\frac{\delta u}{\delta y} = 2xy + \frac{\delta C(y)}{\diff y} = Q = 2xy + y^3$

    $\frac{\delta C}{\delta y} = y^3$

    $\int \diff C = \int y^3 \diff y$

    $C(y) = \frac{y^4}{4} + K$
    \item $u = x^2 + y^2 x + \frac{y^4}{4} + K$
\end{enumerate}

В итоге мы восстановили функцию по ее полному дифференциалу.

\textbf{Ответ}: $x^2 + y^2 x + \frac{y^4}{4} = Const$

Попробуем восстановить ее другим образом:

$\int\limits_{0}^{x} 2x \diff x + \int\limits_{0}^{y} (2xy + y^3) \diff y = \frac{2x^2}{2} \bigg|_{x = 0}^{x} + \frac{2xy^2}{2} \bigg|_{y = 0}^{y} + \frac{y^4}{4} \bigg|_{y = 0}^{y = y}$

$u = x^2 + xy^2 + \frac{y^4}{4} + C$

\textbf{В итоге получили то же самое}.

\paragraph{Пример №3}

$$3y \sin x \diff x - \cos x \diff y = 0$$

$$\frac{\delta P}{\delta y} = 3 \sin x$$

$$\frac{\delta Q}{\delta x} = \sin x$$

Не является уравнением в полных дифференциалах.

\subsubsection{Метод интегрирующего множителя}

Может оказаться, что левая часть уравнения $M(x, y) \diff x + N(x, y) \diff y = 0$ не является полным дифференциалом, то есть, $\frac{\delta M}{\delta y} \ne \frac{\delta N}{\delta x}$. Можно подобрать функцию $\mu (x, y)$ такую, что если мы умножим все это уравнение на $\mu$, то оно окажется полным дифференциалом. Решение полученного уравнения будет совпадать с общим решением исходного.

$$\mu M(x, y) \diff x + \mu N(x, y) \diff y = 0$$

Так как мы хотим, чтобы полученное уравнение оказалось уравнением в полных дифференциалах, мы накладываем требования:

$$\frac{\delta}{\delta x} (\mu N) = \frac{\delta}{\delta y} (\mu M)$$


$$\mu \frac{\delta N}{\delta x} + N \frac{\delta \mu}{\delta x} = \mu \frac{\delta M}{\delta y} + M \frac{\delta \mu}{\delta y}$$

$$\mu (\frac{\delta N}{\delta x} - \frac{\delta M}{\delta y}) = M \frac{\delta M}{\delta y} - N \frac{\delta M}{\delta x}$$

$$\frac{\delta N}{\delta x} - \frac{\delta M}{\delta y} = M * \frac{1}{\mu} \frac{\delta \mu}{\delta y} - N * \frac{1}{\mu} \frac{\delta \mu}{\delta x}$$

$$\frac{\delta N}{\delta x} - \frac{\delta M}{\delta y} = M \frac{\delta (\ln \mu)}{\delta y} - N \frac{\delta (\ln \mu)}{\delta x}$$

Если $\mu$ не зависит от $x$, $\mu = \mu (y)$, то мы можем написать:

$$\frac{\delta (\ln \mu)}{\delta y} = \frac{\frac{\delta N}{\delta x} - \frac{\delta M}{\delta y}}{M}$$

Если $\mu$ не зависит от $y$, $\mu = \mu (x)$, то мы можем написать:


$$\frac{\delta (\ln \mu)}{\delta x} = \frac{\frac{\delta M}{\delta y} - \frac{\delta N}{\delta x}}{N}$$

\paragraph{Примеры}

\paragraph{Пример №1}

$$(y + xy^2) \diff x - x \diff y = 0$$

$$M = (y +xy^2), \ N = -x$$

$$\frac{\delta M}{\delta y} = 2xy + 1, \ \frac{\delta N}{\delta x} = -1$$

Можем записать:

$$\frac{\delta (\ln \mu)}{\delta y} = \frac{-1 - 2xy - 1}{y(1 + xy)} = \frac{-2(xy + 1)}{y (1 + xy)} = -\frac{2}{y}$$


$$\int \diff \ln \mu = - \int \frac{2}{y} \diff y$$

$$\ln \mu = -2 \ln y = \ln \frac{1}{y^2}$$

$$\mu = \frac{1}{y^2}$$

Умножаем наше выражение на $\mu$:

$$\frac{y + xy^2}{y^2} \diff x - \frac{x}{y^2} \diff y = 0$$

$$(\frac{1}{y} + x) \diff x - \frac{x}{y^2} \diff y = 0$$

$$\frac{\delta P}{\delta y} = - \frac{1}{y^2} = \frac{\delta Q}{\delta x} = - \frac{1}{y^2}$$

Можем попытаться восстановить функцию:

$u = \int\limits_{0}^{x} (\frac{1}{y} + x) \diff x + \int\limits_{1}^{y} (0) \diff y = \frac{x}{y} + \frac{x^2}{2} \bigg|_{0}^{x} + C$

Мы восстановили функцию, мы молодцы.

\textbf{Ответ:} $\frac{x}{y} + \frac{x^2}{2} = Const$

Попробуем пойти иначе:

$$(y + xy^2) \diff x - x \diff y = 0$$

$$(y + xy^2) \diff x = x \diff y$$

$$\frac{\diff y}{\diff x} = \frac{y}{x} + y^2$$

$$\frac{\diff y}{\diff x} - \frac{y}{x} = y^2$$

Получили уравнение Бернулли. Решаем дальше:

$$u'v + uv' - \frac{u v}{x} = u^2 v^2$$

$$\frac{\diff v}{\diff x} = \frac{v}{x}$$

$$\int \frac{\diff v}{v} = \int \frac{\diff x}{x}$$

$$v = x$$

Возвращаемся в уравнение:

$$u' * v = u^2 v^2$$

$$\frac{\diff u}{\diff x} = u^2 x$$

$$\int \frac{\diff u}{u^2} = \int x \diff x$$

$$C - \frac{1}{u} = \frac{x^2}{2}$$

$$u = \frac{1}{C - \frac{x^2}{2}}$$

$$y = \frac{x}{C - \frac{x^2}{2}} \Longleftrightarrow \frac{x}{y} = C - \frac{x^2}{2} \Longleftrightarrow \frac{x}{y} + \frac{x^2}{2} = C$$

Получили тот же самый ответ.

\paragraph{Пример №2}

$$\frac{y}{x} \diff x + (y^3 - \ln x) \diff y = 0$$

$$M = \frac{y}{x}, \ N = y^3 - \ln x$$

$$\frac{\delta M}{\delta y} = \frac{1}{x}, \ \frac{\delta N}{\delta x} = - \frac{1}{x}$$

$$\frac{\diff (\ln \mu)}{\diff y} = \frac{\frac{\delta N}{\delta x} - \frac{\delta M}{\delta y}}{\frac{y}{x}} = \frac{-\frac{1}{x} - \frac{1}{x}}{\frac{y}{x}}$$

$\mu$ — не зависит от $x$

$$\int \diff (\ln \mu) = - \int \frac{2}{y} \diff y$$

$$\ln \mu = -2 \ln y$$

$$\mu = \frac{1}{y^2}$$

Умножим все уравнение на $\mu$:

$$\frac{1}{y^2} \frac{y}{x} \diff x + (\frac{y^3}{y^2} + \frac{\ln x}{y^2}) \diff y = 0$$

$$\frac{1}{xy}{\diff x} + (y - \frac{\ln x}{y^2}) \diff y = 0$$

$$u = \int\limits_{1}^{x} \frac{1}{xy} \diff x + \int\limits_{1}^{y} (y ) \diff x + C$$

$$u = \frac{1}{y} \ln x \bigg|_{x = 1}^{x} + \frac{y^2}{2} \bigg|_{y = 1}^{y} + C$$

\textbf{Ответ}: $\frac{1}{y} \ln x + \frac{y^2}{2} = Const$

\subsubsection{Финты ушами}

Допустим, имеем:

$$y = x * \Phi (y') + \Psi (y')$$

$$y' = P$$

$$y = x \Phi (p) + \Psi(p)$$

$$y' = P = 1 * \Phi (p) + x \Phi'(P) * P' + \Psi' (P) P'$$

$$p - \Phi(p) = (x * \Phi'(p) + \Psi'(p)) \frac{\diff p}{\diff x}$$

С этим уравнением мы теперь можем работать.

\subsection{Уравнения высших порядков, допускающие понижение порядка}

Общий вид:

$$F(x, y, y', y'', \dots, y^{(n)}) = 0$$

\begin{enumerate}
    \item $y^{(n)} = f(x)$

    $y^{(n - 1)} = \int f(x) \diff x + C$

    $y^{(n - 2)} = \int F_1(x) \diff x + C_1 x + C$
\end{enumerate}

Например, $y''' = x^2 + 5 \sin x$

\begin{enumerate}
    \item $y'' = \frac{x^3}{3} - 5\cos x + C_1$
    \item $y' = \frac{x^4}{12} - 5 \sin x + C_1 x + C$
    \item $y = \frac{x^5}{60} + 5 \cos x + \frac{C_1 x^2}{2} + C_2 x + C_3$
\end{enumerate}

И задано:

\begin{enumerate}
    \item $y(0) = 2$
    \item $y'(0) = 1$
    \item $y''(0) = 0$
\end{enumerate}

Можем вычислить:

\begin{enumerate}
    \item $y''(0) = \frac{0}{3} - 5 + C_1 = 0$, $C_1 = 5$
    \item $y'(0) = C_2 = 1$
    \item $y(0) = 5 + C_3 = 2 + C_3 = -3$
\end{enumerate}

\textbf{Ответ}: $y = \frac{x^5}{60} + 5 \cos x + \frac{5}{2} x^2 + x - 3$

\end{document}