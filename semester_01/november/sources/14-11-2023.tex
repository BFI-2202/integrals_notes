\documentclass{article}

\usepackage[T2A]{fontenc}
\usepackage[utf8]{inputenc}
\usepackage[russian]{babel}

\usepackage{tabularx}
\usepackage{amsmath}
\usepackage{pgfplots}
\usepackage{geometry}
\usepackage{multicol}
\geometry{
    left=1cm,right=1cm,top=2cm,bottom=2cm
}
\newcommand*\diff{\mathop{}\!\mathrm{d}}

\newtheorem{definition}{Определение}
\newtheorem{theorem}{Теорема}

\DeclareMathOperator{\sign}{sign}

\usepackage{hyperref}
\hypersetup{
    colorlinks, citecolor=black, filecolor=black, linkcolor=black, urlcolor=black
}

\title{Интегралы и дифференциальные уравнения}
\author{Лисид Лаконский}
\date{November 2023}

\begin{document}
\raggedright

\maketitle

\tableofcontents
\pagebreak

\section{Лекция — 14.11.2023}

\subsection{Линейное уравнение и уравнение Бернулли}

Вид линейного уравнения:

$$y' + p(x) y = q(x)$$

Вид уравнения Бернулли:

$$y' + p(x) y = q(x) y^{\alpha}$$

\paragraph{Примеры}

Допустим, имеем

$$\frac{\diff y}{\diff x} = \frac{1}{x \cos y + \sin 2 y}$$

$$\frac{\diff x}{\diff y} = x \cos y + \sin 2 y$$

$$\frac{\diff x}{\diff y} + (\cos y) x = \sin 2y$$

Получившееся уравнение мы можем решать уже знакомыми способами.

$$x = u v, \ x' = u' v + u v'$$

$$u'v + uv' + u v \cos y = \sin 2y$$

\begin{enumerate}
    \item 

    $\frac{\diff v}{\diff y} = -v \cos y$

    $\int \frac{\diff v}{v} = -\cos y \diff y$

    $\ln v = - \sin y$

    $v = e^{-\sin y}$

    \item
    
    $\frac{\diff u}{\diff y} e^{- \sin y} = \sin 2y$

    $\int \diff u = \int e^{\sin y} \sin 2y \diff y$

    $u = 2 \int e^{\sin y} \cos y \diff y = 2 \int e^{\sin y} \sin y \diff (\sin y) = 2 \int t e^{t} \diff t = \begin{vmatrix}
        u = t & \diff v = e^{t} \diff t
        \diff t = \diff u & v = e^{t}
    \end{vmatrix} = 2 (te^{t} - e^{t}) + C$

    $x = uv = 2 (\sin y e^{\sin y} - e^{\sin y} + C) e^{\sin y} = 2 \sin y - 2 + C e^{- \sin y}$
\end{enumerate}

\subsection{Уравнения в полных дифференциалах}

Пусть $u = u(x, y)$, ее полный дифференциал: $\diff u = \frac{\delta u}{\delta x} \diff x + \frac{\delta u}{\delta y} \diff y$

Если $P(x, y) \diff x + Q(x, y) \diff y = 0$, $\frac{\delta u}{\delta x} \diff x + \frac{\delta u}{\delta y}{\diff y} = 0$, то $\diff u = 0 \implies u = C$

$\frac{\delta}{\delta y} (\frac{\delta u}{\delta x}) = \frac{\delta}{\delta x} (\frac{\delta u}{\delta y})$

Если $\frac{\delta P}{\delta y} = \frac{\delta Q}{\delta x}$, то имеем уравнение в полных дифференциалах

Если мы уверены, что уравнение — уравнение в полных дифференциалах, то мы можем сделать так:

$\diff u = P(x, y) \diff x + Q(x, y) \diff y$

$u = \int\limits_{x_0}^{x} P(x, y_0) \diff x + \int\limits_{y_0}^{y} Q(x, y) \diff y + C = \int\limits_{x_0}^{x} P(x; y) \diff x + \int\limits_{y_0}^{y} Q(x_0; y) \diff y + C$

$x_0$, $y_0$ — координаты некоторой точки, которая находится в области определения функций $P$ и $Q$, частные производные которых в этой точке также непрерывны.

\paragraph{Примеры}

\paragraph{Пример №1}

$$(2xy + y^2) \diff x + (3x^2 + y) \diff y = 0$$

$$\frac{\delta P}{\delta y} = 2x + 2y$$

$$\frac{\delta Q}{\delta x} = 6x$$

Не является уравнением в полных дифференциалах.

\paragraph{Пример №2}

$$(2x + y^2) \diff x + (2xy + y^3) \diff y = 0$$

$$\frac{\delta P}{\delta y} = 2y$$

$$\frac{\delta Q}{\delta x} = 2y$$

Является уравнением в полных дифференциалах.

\begin{enumerate}
    \item
    
    $\frac{\delta u}{\delta x} = 2x + y^2$

    $u = \int (2x + y^2) \diff x = x^2 + y^2 x + C(y)$
    \item 

    $\frac{\delta u}{\delta y} = 2xy + \frac{\delta C(y)}{\diff y} = Q = 2xy + y^3$

    $\frac{\delta C}{\delta y} = y^3$

    $\int \diff C = \int y^3 \diff y$

    $C(y) = \frac{y^4}{4} + K$
    \item $u = x^2 + y^2 x + \frac{y^4}{4} + K$
\end{enumerate}

В итоге мы восстановили функцию по ее полному дифференциалу.

\textbf{Ответ}: $x^2 + y^2 x + \frac{y^4}{4} = Const$

Попробуем восстановить ее другим образом:

$\int\limits_{0}^{x} 2x \diff x + \int\limits_{0}^{y} (2xy + y^3) \diff y = \frac{2x^2}{2} \bigg|_{x = 0}^{x} + \frac{2xy^2}{2} \bigg|_{y = 0}^{y} + \frac{y^4}{4} \bigg|_{y = 0}^{y = y}$

$u = x^2 + xy^2 + \frac{y^4}{4} + C$

\textbf{В итоге получили то же самое}.

\paragraph{Пример №3}

$$3y \sin x \diff x - \cos x \diff y = 0$$

$$\frac{\delta P}{\delta y} = 3 \sin x$$

$$\frac{\delta Q}{\delta x} = \sin x$$

Не является уравнением в полных дифференциалах.

\subsubsection{Метод интегрирующего множителя}

Может оказаться, что левая часть уравнения $M(x, y) \diff x + N(x, y) \diff y = 0$ не является полным дифференциалом, то есть, $\frac{\delta M}{\delta y} \ne \frac{\delta N}{\delta x}$. Можно подобрать функцию $\mu (x, y)$ такую, что если мы умножим все это уравнение на $\mu$, то оно окажется полным дифференциалом. Решение полученного уравнения будет совпадать с общим решением исходного.

$$\mu M(x, y) \diff x + \mu N(x, y) \diff y = 0$$

Так как мы хотим, чтобы полученное уравнение оказалось уравнением в полных дифференциалах, мы накладываем требования:

$$\frac{\delta}{\delta x} (\mu N) = \frac{\delta}{\delta y} (\mu M)$$


$$\mu \frac{\delta N}{\delta x} + N \frac{\delta \mu}{\delta x} = \mu \frac{\delta M}{\delta y} + M \frac{\delta \mu}{\delta y}$$

$$\mu (\frac{\delta N}{\delta x} - \frac{\delta M}{\delta y}) = M \frac{\delta M}{\delta y} - N \frac{\delta M}{\delta x}$$

$$\frac{\delta N}{\delta x} - \frac{\delta M}{\delta y} = M * \frac{1}{\mu} \frac{\delta \mu}{\delta y} - N * \frac{1}{\mu} \frac{\delta \mu}{\delta x}$$

$$\frac{\delta N}{\delta x} - \frac{\delta M}{\delta y} = M \frac{\delta (\ln \mu)}{\delta y} - N \frac{\delta (\ln \mu)}{\delta x}$$

Если $\mu$ не зависит от $x$, $\mu = \mu (y)$, то мы можем написать:

$$\frac{\delta (\ln \mu)}{\delta y} = \frac{\frac{\delta N}{\delta x} - \frac{\delta M}{\delta y}}{M}$$

Если $\mu$ не зависит от $y$, $\mu = \mu (x)$, то мы можем написать:


$$\frac{\delta (\ln \mu)}{\delta x} = \frac{\frac{\delta M}{\delta y} - \frac{\delta N}{\delta x}}{N}$$

\paragraph{Примеры}

\paragraph{Пример №1}

$$(y + xy^2) \diff x - x \diff y = 0$$

$$M = (y +xy^2), \ N = -x$$

$$\frac{\delta M}{\delta y} = 2xy + 1, \ \frac{\delta N}{\delta x} = -1$$

Можем записать:

$$\frac{\delta (\ln \mu)}{\delta y} = \frac{-1 - 2xy - 1}{y(1 + xy)} = \frac{-2(xy + 1)}{y (1 + xy)} = -\frac{2}{y}$$


$$\int \diff \ln \mu = - \int \frac{2}{y} \diff y$$

$$\ln \mu = -2 \ln y = \ln \frac{1}{y^2}$$

$$\mu = \frac{1}{y^2}$$

Умножаем наше выражение на $\mu$:

$$\frac{y + xy^2}{y^2} \diff x - \frac{x}{y^2} \diff y = 0$$

$$(\frac{1}{y} + x) \diff x - \frac{x}{y^2} \diff y = 0$$

$$\frac{\delta P}{\delta y} = - \frac{1}{y^2} = \frac{\delta Q}{\delta x} = - \frac{1}{y^2}$$

Можем попытаться восстановить функцию:

$u = \int\limits_{0}^{x} (\frac{1}{y} + x) \diff x + \int\limits_{1}^{y} (0) \diff y = \frac{x}{y} + \frac{x^2}{2} \bigg|_{0}^{x} + C$

Мы восстановили функцию, мы молодцы.

\textbf{Ответ:} $\frac{x}{y} + \frac{x^2}{2} = Const$

Попробуем пойти иначе:

$$(y + xy^2) \diff x - x \diff y = 0$$

$$(y + xy^2) \diff x = x \diff y$$

$$\frac{\diff y}{\diff x} = \frac{y}{x} + y^2$$

$$\frac{\diff y}{\diff x} - \frac{y}{x} = y^2$$

Получили уравнение Бернулли. Решаем дальше:

$$u'v + uv' - \frac{u v}{x} = u^2 v^2$$

$$\frac{\diff v}{\diff x} = \frac{v}{x}$$

$$\int \frac{\diff v}{v} = \int \frac{\diff x}{x}$$

$$v = x$$

Возвращаемся в уравнение:

$$u' * v = u^2 v^2$$

$$\frac{\diff u}{\diff x} = u^2 x$$

$$\int \frac{\diff u}{u^2} = \int x \diff x$$

$$C - \frac{1}{u} = \frac{x^2}{2}$$

$$u = \frac{1}{C - \frac{x^2}{2}}$$

$$y = \frac{x}{C - \frac{x^2}{2}} \Longleftrightarrow \frac{x}{y} = C - \frac{x^2}{2} \Longleftrightarrow \frac{x}{y} + \frac{x^2}{2} = C$$

Получили тот же самый ответ.

\paragraph{Пример №2}

$$\frac{y}{x} \diff x + (y^3 - \ln x) \diff y = 0$$

$$M = \frac{y}{x}, \ N = y^3 - \ln x$$

$$\frac{\delta M}{\delta y} = \frac{1}{x}, \ \frac{\delta N}{\delta x} = - \frac{1}{x}$$

$$\frac{\diff (\ln \mu)}{\diff y} = \frac{\frac{\delta N}{\delta x} - \frac{\delta M}{\delta y}}{\frac{y}{x}} = \frac{-\frac{1}{x} - \frac{1}{x}}{\frac{y}{x}}$$

$\mu$ — не зависит от $x$

$$\int \diff (\ln \mu) = - \int \frac{2}{y} \diff y$$

$$\ln \mu = -2 \ln y$$

$$\mu = \frac{1}{y^2}$$

Умножим все уравнение на $\mu$:

$$\frac{1}{y^2} \frac{y}{x} \diff x + (\frac{y^3}{y^2} + \frac{\ln x}{y^2}) \diff y = 0$$

$$\frac{1}{xy}{\diff x} + (y - \frac{\ln x}{y^2}) \diff y = 0$$

$$u = \int\limits_{1}^{x} \frac{1}{xy} \diff x + \int\limits_{1}^{y} (y ) \diff x + C$$

$$u = \frac{1}{y} \ln x \bigg|_{x = 1}^{x} + \frac{y^2}{2} \bigg|_{y = 1}^{y} + C$$

\textbf{Ответ}: $\frac{1}{y} \ln x + \frac{y^2}{2} = Const$

\subsubsection{Финты ушами}

Допустим, имеем:

$$y = x * \Phi (y') + \Psi (y')$$

$$y' = P$$

$$y = x \Phi (p) + \Psi(p)$$

$$y' = P = 1 * \Phi (p) + x \Phi'(P) * P' + \Psi' (P) P'$$

$$p - \Phi(p) = (x * \Phi'(p) + \Psi'(p)) \frac{\diff p}{\diff x}$$

С этим уравнением мы теперь можем работать.

\subsection{Уравнения высших порядков, допускающие понижение порядка}

Общий вид:

$$F(x, y, y', y'', \dots, y^{(n)}) = 0$$

\begin{enumerate}
    \item $y^{(n)} = f(x)$

    $y^{(n - 1)} = \int f(x) \diff x + C$

    $y^{(n - 2)} = \int F_1(x) \diff x + C_1 x + C$
\end{enumerate}

Например, $y''' = x^2 + 5 \sin x$

\begin{enumerate}
    \item $y'' = \frac{x^3}{3} - 5\cos x + C_1$
    \item $y' = \frac{x^4}{12} - 5 \sin x + C_1 x + C$
    \item $y = \frac{x^5}{60} + 5 \cos x + \frac{C_1 x^2}{2} + C_2 x + C_3$
\end{enumerate}

И задано:

\begin{enumerate}
    \item $y(0) = 2$
    \item $y'(0) = 1$
    \item $y''(0) = 0$
\end{enumerate}

Можем вычислить:

\begin{enumerate}
    \item $y''(0) = \frac{0}{3} - 5 + C_1 = 0$, $C_1 = 5$
    \item $y'(0) = C_2 = 1$
    \item $y(0) = 5 + C_3 = 2 + C_3 = -3$
\end{enumerate}

\textbf{Ответ}: $y = \frac{x^5}{60} + 5 \cos x + \frac{5}{2} x^2 + x - 3$

\end{document}