\documentclass{article}

\usepackage[T2A]{fontenc}
\usepackage[utf8]{inputenc}
\usepackage[russian]{babel}

\usepackage{tabularx}
\usepackage{amsmath}
\usepackage{pgfplots}
\usepackage{geometry}
\usepackage{multicol}
\geometry{
    left=1cm,right=1cm,top=2cm,bottom=2cm
}
\newcommand*\diff{\mathop{}\!\mathrm{d}}

\newtheorem{definition}{Определение}
\newtheorem{theorem}{Теорема}

\DeclareMathOperator{\sign}{sign}

\usepackage{hyperref}
\hypersetup{
    colorlinks, citecolor=black, filecolor=black, linkcolor=black, urlcolor=black
}

\title{Интегралы и дифференциальные уравнения}
\author{Лисид Лаконский}
\date{September 2023}

\begin{document}
\raggedright

\maketitle

\tableofcontents
\pagebreak

\section{Домашнее задание — 07.09.2023}

\subsection{№ 3492}

Найти пределы двукратного интеграла $\int\int\limits_{D} f(x, y) \diff x \diff y$ при данных (конечных) областях интегрирования $D$: $D$ ограничена параболами $y=x^2$ и $y = \sqrt{x}$

\textbf{График}:

\begin{tikzpicture}
\begin{axis}[
    axis lines = left,
    xlabel = \(x\),
    ylabel = {\(f(x)\)},
]
\addplot [
    domain=-2:2, 
    samples=200, 
    color=red,
]
{x^2};
\addlegendentry{\(x^2\)}
\addplot [
    domain=-2:2, 
    samples=200, 
    color=blue,
    ]
    {sqrt(x)};
\addlegendentry{\(sqrt(x)\)}
\end{axis}
\end{tikzpicture}

\textbf{Решение}:

$\int\int\limits_{D} f(x, y) \diff x \diff y = \int\limits_{0}^{1} \diff x \int\limits_{0}^{\sqrt{x}} f(x, y) \diff y$

$\int\int\limits_{D} f(x, y) \diff x \diff y = \int\limits_{0}^{1} \diff y \int\limits_{0}^{x^2} f(x, y) \diff x$

\subsection{№ 3493}

Найти пределы двукратного интеграла $\int\int\limits_{D} f(x, y) \diff x \diff y$ при данных (конечных) областях интегрирования $D$: треугольник со сторонами $y = x$, $y = 2x$ и $x + y = 6$

\textbf{График}:

\begin{tikzpicture}
\begin{axis}[
    axis lines = left,
    xlabel = \(x\),
    ylabel = {\(f(x)\)},
    xtick distance=0.5,
    ytick distance=1,
]
\addplot [
    domain=-2:5, 
    samples=1000, 
    color=red,
]
{x};
\addlegendentry{\(x\)}
\addplot [
    domain=-2:5, 
    samples=1000, 
    color=green,
    ]
    {2*x};
\addlegendentry{\(2x\)}
\addplot [
    domain=-2:5, 
    samples=1000, 
    color=blue,
    ]
    {6-x};
\addlegendentry{\(6-x\)}
\end{axis}
\end{tikzpicture}

\textbf{Решение}:

$\int\int\limits_{D} f(x, y) \diff x \diff y = \int\limits_{0}^{2} \diff x \int\limits_{0}^{2x} f(x, y) \diff y + \int\limits_{2}^{3} \diff x \int\limits_{2}^{6-x} f(x, y) \diff y$

$\int\int\limits_{D} f(x, y) \diff x \diff y = \int\limits_{0}^{3} \diff y \int\limits_{0}^{x} f(x, y) \diff x + \int\limits_{3}^{4} \diff y \int\limits_{1.5}^{6-x} f(x, y) \diff x$

\subsection{№ 3494}

Найти пределы двукратного интеграла $\int\int\limits_{D} f(x, y) \diff x \diff y$ при данных (конечных) областях интегрирования $D$: параллелограмм со сторонами $y = x$, $y = x + 3$, $y = -2x + 1$, $y = -2x + 5$

\textbf{График:}

\begin{tikzpicture}
\begin{axis}[
    axis lines = left,
    xlabel = \(x\),
    ylabel = {\(f(x)\)},
]
\addplot [
    domain=-1:2, 
    samples=1000, 
    color=red,
]
{x};
\addlegendentry{\(x\)}
\addplot [
    domain=-1:2, 
    samples=1000, 
    color=green,
]
{x+3};
\addlegendentry{\(x+3\)}
\addplot [
    domain=-1:2, 
    samples=1000, 
    color=blue,
]
{-2*x+1};
\addlegendentry{\(-2x+1\)}
\addplot [
    domain=-1:2, 
    samples=1000, 
    color=black,
]
{-2*x+5};
\addlegendentry{\(-2x+5\)}
\end{axis}
\end{tikzpicture}

\textbf{Решение:}

Найдем точку пересечения второй и третьей линий: $x + 3 = -2x + 1 \Longleftrightarrow 3x = -2 \Longleftrightarrow x = -\frac{2}{3}, y = \frac{7}{3}$

Найдем точку пересечения первой и третьей линий: $x = -2x + 1 \Longleftrightarrow 3x = 1 \Longleftrightarrow x = \frac{1}{3}, y = \frac{1}{3}$

Найдем точку пересечения второй и четвертой линий: $x + 3 = -2x + 5 \Longleftrightarrow 3x = 2 \Longleftrightarrow x = \frac{2}{3}, y = \frac{11}{3}$

Найдем точку пересечения первой и четвертой линий: $x = -2x + 5 \Longleftrightarrow 3x = 5 \Longleftrightarrow x = \frac{5}{3}, y = \frac{5}{3}$

$\int\int\limits_{D} f(x, y) \diff x \diff y = \int\limits_{-\frac{2}{3}}^{\frac{1}{3}} \diff x \int\limits_{-2x+1}^{x+3} f(x, y) \diff y + \int\limits_{\frac{1}{3}}^{\frac{2}{3}} \diff x \int\limits_{x}^{x+3} f(x, y) \diff y + \int\limits_{\frac{2}{3}}^{\frac{5}{3}} \diff x \int\limits_{x}^{-2x+5} f(x, y) \diff y$

$\int\int\limits_{D} f(x, y) \diff x \diff y = \int\limits_{\frac{1}{3}}^{\frac{5}{3}} \diff y \int\limits_{-2x+1}^{x} f(x, y) \diff x + \int\limits_{\frac{5}{3}}^{\frac{7}{3}} \diff y \int\limits_{-2x+1}^{-2x+5} f(x, y) \diff x + \int\limits_{\frac{7}{3}}^{\frac{11}{3}} \diff y \int\limits_{x+3}^{-2x+5} f(x, y) \diff x$

\subsection{№ 3506}

Вычислить данные интегралы:

\begin{enumerate}
    \item $\int \limits_{0}^{a} \diff x \int\limits_{0}^{\sqrt{x}} \diff y = \int\limits_{0}^{a} (\sqrt{x} - 0) \diff x = \frac{2x^{\frac{3}{2}}}{3} \bigg|_{0}^{a} = \frac{2a^{\frac{3}{2}}}{3}$
    \item $\int\limits_{2}^{4} \diff x \int\limits_{x}^{2x} \frac{y}{x} \diff y = \int\limits_{2}^{4} (\frac{1}{x} (\frac{y^2}{2})\bigg|_{x}^{2x}) \diff x = \int\limits_{2}^{4} (\frac{1}{x} * \frac{3x^2}{2}) \diff x = \int\limits_{2}^{4} \frac{3x^2}{2x} \diff x = (\frac{3x^2}{4}) \bigg|_{2}^{4} = \frac{3*16}{4} - \frac{3 * 4}{4} = 12 - 3 = 9$
    \item $\int\limits_{1}^{2} \diff y \int\limits_{0}^{\ln x} e^{x} \diff x = \int\limits_{1}^{2} ((e^{x}) \bigg|_{0}^{\ln x}) \diff y = \int\limits_{1}^{2} (e^{\ln x} - 1) \diff y = \int\limits_{1}^{2} (x - 1) \diff y = (xy - y) \bigg|_{1}^{2} = (2x - 2) - (x - 1) = x - 1$
\end{enumerate}

\subsection{№ 3508}

Вычислить данный интеграл: $\int\int\limits_{D} (x^2 + y) \diff x \diff y$, где $D$ — область, ограниченная параболами $y = x^2$ и $y^2 = x$

\textbf{Решение:}

$\int\int\limits_{D} (x^2 + y) \diff x \diff y = \int\limits_{0}^{1} \diff x \int\limits_{x^2}^{\sqrt{x}} (x^2 + y) \diff y = \int\limits_{0}^{1} ((x^2 y + \frac{y^2}{2}) \bigg|_{x^2}^{\sqrt{x}}) \diff x = \int\limits_{0}^{1} ((x^{2.5} + \frac{x}{2}) - (x^4 + \frac{x^4}{2})) \diff x = \int\limits_{0}^{1} (x^{2.5} + \frac{1}{2} x - \frac{1}{2} x^4) \diff x = \int\limits_{0}^{1} (x^{5/2}) \diff x + \frac{1}{2} \int\limits_{0}^{1} (x) - \frac{1}{2} \int\limits_{0}^{1} (x^4) \diff x = 0.435714$

\end{document}