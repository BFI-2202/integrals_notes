\documentclass{article}

\usepackage[T2A]{fontenc}
\usepackage[utf8]{inputenc}
\usepackage[russian]{babel}

\usepackage{tabularx}
\usepackage{amsmath}
\usepackage{pgfplots}
\usepackage{geometry}
\usepackage{multicol}
\geometry{
    left=1cm,right=1cm,top=2cm,bottom=2cm
}
\newcommand*\diff{\mathop{}\!\mathrm{d}}

\newtheorem{definition}{Определение}
\newtheorem{theorem}{Теорема}

\DeclareMathOperator{\sign}{sign}

\usepackage{hyperref}
\hypersetup{
    colorlinks, citecolor=black, filecolor=black, linkcolor=black, urlcolor=black
}

\title{Интегралы и дифференциальные уравнения}
\author{Лисид Лаконский}
\date{September 2023}

\begin{document}
\raggedright

\maketitle

\tableofcontents
\pagebreak

\section{Практическое занятие — 21.09.2023}

\subsection{Переход к полярным координатам}

\subsubsection{№3525}

\paragraph{1)} $x^2 + y^2 \le R^2$

Выполним построение. После чего выполним переход к полярным координатам:

$\rho^2 \cos^2 \phi + \rho^2 \sin^2 \phi =  R^2$

$\rho^2 = R^2$

$\rho = R$

$\int\limits_{0}^{2 \pi} \diff \phi \int\limits_{0}^{R} f(\rho \cos \phi, \rho \sin \phi) \rho \diff \rho$

\paragraph{2)}

$x^2+y^2 \le 2x$

$(x^2-2x+1) + y^2 \le 1$

$(x-1)^2 + y^2 \le 1$

Выполним построение. После чего выполним переход к полярным координатам:

$\rho^2 \cos^2 \phi + \rho^2 \sin^2 \phi = 2 \rho \cos \phi$

$\rho^2 = 2 \rho \cos \phi$

$\rho = 2 \cos \phi$

Так что если мы вычисляем двойной интеграл, то делаем следующее:

$\int\int\limits_{D} f(x, y) \diff x \diff y = \int\limits_{-\frac{\pi}{2}}^{\frac{\pi}{2}} \diff \phi \int\limits_{0}^{2 \cos \phi} f(\rho \cos \phi, \rho \sin \phi) * \rho \diff \rho$

\paragraph{3)}

$x^2 + y^2 \le 4y$

$x^2 + y^2 - 4y + 4 \le 4$

$x^2 + (y - 2)^2 \le 2^2$

Выполним построение. После чего выполним переход к полярным координатам:

$\rho^2 \cos^2 \phi + \rho^2 \sin^2 \phi = 4 \rho \sin \phi$

$\rho = 4 \sin \phi$

$\int\limits_{0}^{\pi} \diff \phi \int\limits_{0}^{4 \sin \phi}(\rho \cos \phi, \rho \sin \phi) \rho \diff \rho$

\subsubsection{№3526}

$D$ — область, ограниченная окружностями $x^2 + y^2 = 4x$, $x^2 + y^2 = 8x$ и прямыми $y = x$ и $y = 2x$.

Перейти в двойном интеграле $\int\int\limits_{D} f(x, y) \diff x \diff y$ к полярным координатам $\rho$ и $\phi$ ($x = \rho \cos \phi, y = \rho \sin \phi$) и расставить пределы интегрирования.

\begin{tikzpicture}
\begin{axis}[
    axis lines = left,
    xlabel = \(x\),
    ylabel = {\(f(x)\)},
]
\addplot [
    domain=-2:2, 
    samples=200, 
    color=red,
]
{sqrt(4*x-x^2)};
\addplot [
    domain=-2:2, 
    samples=200, 
    color=blue,
    ]
    {sqrt(8*x-x^2)};
\addplot [
    domain=-2:2, 
    samples=200, 
    color=green,
    ]
    {x};
\addplot [
    domain=-2:2, 
    samples=200, 
    color=black,
    ]
    {2*x};
\end{axis}
\end{tikzpicture}

$\rho^2 \cos^2 \phi + \rho^2 \sin^2 \phi = 4 \rho \cos \phi$

$\rho = 4 \cos \phi$

$\rho^2 \cos^2 \phi + \rho^2 \sin^2 \phi = 8 \rho \cos \phi$

$\rho = 8 \cos \phi$

$\phi \cos \phi = \phi \sin \phi$, $\tan \phi = 1$, $\phi = \frac{\pi}{4}$

$8 \sin \phi = 2 \phi \cos \phi$, $\tan \phi = \arctan 2$

$\int\int\limits_{D} f(x, y) \diff x \diff y = \int\limits_{\frac{\pi}{4}}^{\arctan 2} \diff \phi \int\limits_{4 \cos \phi}^{8 \cos \phi} f(\rho \cos \phi; \rho \sin \phi) \rho \diff \rho$

\subsubsection{№3527}

Перейти в двойном интеграле $\int\int\limits_{D} f(x, y) \diff x \diff y$ к полярным координатам $\rho$ и $\phi$ ($x = \rho \cos \phi, y = \rho \sin \phi$) и расставить пределы интегрирования.

$x^2 + y^2 \le 2x$, $x^2 + y^2 \le 4y$

$x^2 - 2x + 1 + y^2 \le 1$, $x^2 + y^2 - 4y + 4 \le 4$

$(x-1)^2 + y^2 \le 1$, $x^2 + (y-2)^2 \le 4$

Выполним построение. После чего выполним переход к полярным координатам:

$\rho^2 \cos^2 \phi + \rho^2 \sin^2 \phi = 2 \rho \cos \phi$

$\rho = 2 \cos \phi$

$\rho^2 \cos^2 \phi + \rho^2 \sin^2 \phi \le 4 \rho \sin \phi$

$\rho = 4 \sin \phi$

Найдем угол пересечения:

$2 \rho \cos \phi = 4 \rho \sin \phi$

$2 = 4 \tan \phi$

$\tan \phi = \frac{1}{2}$

Таким образом:

$\int\int\limits_{D} f(x, y) \diff x \diff y = \int\limits_{0}^{\arctan \frac{1}{2}} \diff \phi \int\limits_{0}^{4 \sin \phi} f(\rho \cos \phi; \rho \sin \phi) \rho \diff \rho + \int\limits_{\arctan \frac{1}{2}}^{\frac{\pi}{2}} \diff \phi \int\limits_{0}^{2 \cos \phi} f(\rho \cos \phi; \rho \sin \phi) \rho \diff \rho = \dots$

\subsubsection{№3528}

Перейти в двойном интеграле $\int\int\limits_{D} f(x, y) \diff x \diff y$ к полярным координатам $\rho$ и $\phi$ ($x = \rho \cos \phi, y = \rho \sin \phi$) и расставить пределы интегрирования.

$y=x$

$y=0$

$x=1$

Выполним построение. После чего выполним переход к полярным координатам:

$x = 1 \implies \rho \cos \phi = 1 \implies \rho = \frac{1}{\cos \phi}$

$\int\limits_{0}^{\frac{\pi}{4}} \diff \phi \int\limits_{0}^{\frac{1}{\cos \phi}} f(\rho \cos \phi; \rho \sin \phi) \rho \diff \rho$

\subsubsection{№3529}

Перейти в двойном интеграле $\int\int\limits_{D} f(x, y) \diff x \diff y$ к полярным координатам $\rho$ и $\phi$ ($x = \rho \cos \phi, y = \rho \sin \phi$) и расставить пределы интегрирования.

Прямая $x + y = 2$ рассекает круг $x^2 + y^2 \le 4$.

$\rho \cos \phi + \rho \sin \phi = 2$

$\rho (\cos \phi + \sin \phi) = 2$

$\rho = \frac{2}{\cos \phi + \sin \phi}$

$\int\limits_{0}^{\frac{\pi}{2}} \diff \phi \int\limits_{\frac{2}{\cos \phi + \sin \phi}}^{2} f(\rho \cos \phi; \rho \sin \phi) \rho \diff \rho$

\subsubsection{№3536}

С помощью перехода к полярным координатам вычислить двойной интеграл:

$\int\limits_{0}^{R} \diff x \int\limits_{0}^{\sqrt{R^2-x^2}} \ln (1 + x^2 + y^2) \diff y \ dots$

$y^2 = (\sqrt{R^2-x^2})^2$

$y^2 = R^2 - x^2$

$x^2 + y^2 = R^2$

Выполним построение данной окружности, после чего выполним переход к полярным координатам:

$\rho^2 \cos^2 \phi + \rho^2 \sin^2 \phi = R^2$

$\rho^2 = R^2$, $\rho = R$

$\dots = \int\limits_{0}^{\frac{\pi}{2}} \diff \phi \int\limits_{0}^{R} \ln (1 + \rho^2\cos^2 \phi + \rho^2 \sin^2 \phi) \rho \diff \rho = \int\limits_{0}^{\frac{\pi}{2}} \diff \phi \int\limits_{0}^{R} \ln (1 + \rho^2) \rho \diff \rho = \dots$

$\int\limits_{0}^{R} \ln (1 + \rho^2) \rho \diff \rho = \dots$

$\diff (1 + \rho^2) = 2\rho \diff \rho$

$\dots = \frac{1}{2} \int\limits_{0}^{R} \ln (1 + \rho^2) \diff (1 + \rho^2) = \dots$

Пусть $t = 1 + \rho^2$, $r = 0 \implies t = 1$, $r = \rho \implies t = 1 + R^2$

$\dots = \frac{1}{2} \int\limits_{1}^{1 + R^2} \ln t \diff t = \dots$

Интегрируем по частям: $u = \ln t$, $\diff v = \diff t$, $v = t$, $\diff u = \frac{\diff t}{t}$

$\dots = \frac{1}{2} (t \ln t - t) \bigg|_{1}^{1 + R^2} = \frac{1}{2} ((1 + R^2)(\ln(1+R^2) - 1) + 1) = \frac{\pi}{4} (\ln(1+R^2) + R^2 \ln(1 + R^2) - R^2)$

$\dots = \int\limits_{0}^{\frac{\pi}{2}} (\frac{\pi}{4} (\ln(1+R^2) + R^2 \ln(1 + R^2) - R^2)) \diff \phi $

\subsubsection{№3537}

С помощью перехода к полярным координатам вычислить двойной интеграл:

$\int\int\limits_{D} \sqrt{\frac{1 - x^2 - y^2}{1 + x^2 + y^2}} \diff x \diff y$, где область $D$ определяется неравенствами $x^2 + y^2 \le 1$, $x \ge 0$, $y \ge 0$

$\rho^2 \cos^2 \phi + \rho^2 \sin^2 \phi = \rho^2$

$\rho^2 = 1$

$\rho = 1$

$\int\int\limits_{D} \sqrt{\frac{1 - x^2 - y^2}{1 + x^2 + y^2}} \diff x \diff y = \int\limits_{}^{} \diff \phi \int\limits_{}^{} \sqrt{\frac{1 - x^2 - y^2}{1 + x^2 + y^2}} \rho \diff \rho = \int\limits_{0}^{\frac{\pi}{2}} \diff \phi \int\limits_{0}^{1} \sqrt{\frac{1 - \rho^2}{1 + \rho^2}} \rho \diff \rho$

$\diff \rho^2 = 2 \rho \diff \rho$

$\int\limits_{0}^{1} \sqrt{\frac{1-\rho^2}{1+\rho^2}} \rho \diff \rho = \frac{1}{2} \int\limits_{0}^{1} \sqrt{\frac{1 - \rho^2}{1 + \rho^2}} \diff \rho^2$

Пусть $t = \rho^2$

$\int\limits_{0}^{1} \sqrt{\frac{1 - t}{1 + t}} \diff t$

Пусть $z = \sqrt{\frac{1 - t}{1 + t}}$

$\frac{1-t}{1+t} = t^2$, $z^2 + 1 + t^2 = 1 - t$, $t(z^2 + 1) = 1 - z^2$, $t = \frac{1 - z^2}{z^2 + 1}$

$\diff t = \frac{-2z (z^2 + 1) - 2z (1 - z^2)}{(z^2 + 1)^2} \diff z = -\frac{4z \diff z}{(z^2 + 1)^2}$

$-\frac{1}{2} \int z \frac{4z \diff z}{(z^2+1)^2} = -2 \int \frac{z^2 + 1 -1 \diff z}{(z^2 + 1)^2}$

Дальше должно будет посчитаться до чего-то, мы пихнем в интеграл, и все будет хорошо.

\subsubsection{№3540}

С помощью перехода к полярным координатам вычислить двойной интеграл:

$\int\int\limits_{D} \arctan \frac{y}{x} \diff x \diff y$, где $D$ — часть кольца $x^2 + y^2 \ge 1$, $x^2 + y^2 \le 9$, $y \ge \frac{x}{\sqrt{x}}$, $y \le x\sqrt{3}$

Выполним построение данной окружности, после чего выполним переход к полярным координатам:

$\rho \sin \phi = \frac{\rho \cos \phi}{3}$

$\sqrt{3} \sin \phi = \cos \phi$

$\tan \phi = \frac{1}{\sqrt{3}} \implies \phi = \frac{\pi}{6}$

$y \ge \frac{x}{\sqrt{x}} \implies \phi = \frac{\pi}{6}$

$y \le x\sqrt{3} \implies \phi = \frac{\pi}{3}$

$\arctan \frac{y}{x} = \arctan \frac{\rho \sin \phi}{\rho \cos \phi} = \arctan (\tan \phi) = \phi$

$\int\limits_{\frac{\pi}{6}}^{\frac{\pi}{3}} \phi \diff \phi \int\limits_{1}^{3} \rho \diff \rho =
\int\limits_{\frac{\pi}{6}}^{\frac{\pi}{3}} \phi * (\frac{\rho^2}{2}) \bigg|_{1}^{3} \diff phi = 4 \int\limits_{\pi/6}^{\pi/3} \phi \diff \phi = 2 \phi^2 \bigg|_{\pi/6}^{\pi/3} = 2 (\frac{\pi^2}{9} - \frac{\pi^2}{36}) = 2 (\frac{4\pi^2 - \pi^2}{36}) = \frac{\pi}{6}$

\subsubsection{3548}

Область, ограниченная цилиндром $x^2+y^2=2x$, плоскостью $z = 0$ и параболоидом $z = x^2 + y^2$.

Выполним построение. Нарисуем цилиндр, сдвинутый на 1 ед. о. по $x$, параболоид. Они где-то пересекаются — мы должны расставить пределы по этой области.

Если бы мы работали только в декартовых координатах, то мы бы могли расписать следующим образом:

$\int\limits_{0}^{2} \diff x \int\limits_{-\sqrt{2x-x^2}}^{\sqrt{2x-x^2}} \diff y \int\limits_{0}^{x^2+y^2} f(x, y, z) \diff z$

Если бы мы захотели перейти к полярным координатам:

$x = \rho \cos \phi$, $y = \rho \sin \phi$, $z = z$

$\int\limits_{-\pi/2}^{\pi/2} \diff \phi \int\limits_{0}^{2 \cos \phi} \rho \diff \rho \int\limits_{0}^{\rho^2} f(\rho \cos \pi, \rho \sin \phi, z) \diff z$

\pagebreak
\subsection{Переход к цилиндрическим и сферическим координатам}

\subsubsection{3554}

$\int\limits_{-R}^{R} \diff x \int\limits_{-\sqrt{R^2-x^2}}^{\sqrt{R^2-x^2}} \diff y \int\limits_{0}^{\sqrt{R^2-x^2-y^2}} (x^2 + y^2) \diff z$

Выполним построение: у нас должна получиться половинка сферы. Перейдем в цилиндрические координаты:

$
\int\limits_{0}^{2\pi} \diff \phi \int\limits_{0}^{R} \rho \diff \rho \int\limits_{0}^{\sqrt{R^2-\rho^2}} (\rho)^2 \diff z =
\int\limits_{0}^{2\pi} \diff \phi \int\limits_{0}^{R} \rho^3 \diff \rho \int\limits_{0}^{\sqrt{R^2-\rho^2}} \diff z = 2 \pi \int\limits_{0}^{R} \rho^2 * \rho \sqrt{R^2 - \rho^2} \diff \rho = \pi \int\limits_{0}^{R} \rho^2 \sqrt{R^2 - \rho^2} \diff \rho^2
= -\pi \int\limits_{0}^{R} -\rho^2+R^2-R^2 \sqrt{R^2 - \rho^2} \diff \rho^2 =
-\pi \int\limits (-\rho^2 + R^2) \sqrt{R^2 - \rho^2} \diff \rho^2 + \pi R^2 \int \sqrt{R^2 - \rho^2} \diff \rho^2
= \pi \int (-\rho^2 + R^2) \diff (R^2 - \rho^2) - \pi R^2 \int \sqrt{R^2 - \rho^2} \diff (R^2 - \rho^2)
= 2 \pi (\frac{(R^2 - \rho^2)}{5})^{5/2} - 2 \pi R^2 ()\frac{(R^2 - \rho^2)}{3})^{3/2} \bigg|_{0}^{R} = - \frac{2 \pi R^{2}}{5}^{5/8} + \frac{2 \pi R^2}{3}^{3/8} = 2 \pi R^{5} (\frac{1}{3} - \frac{1}{5})
$

Попробуем выполнить переход к цилиндрическим координатам:

$x = \rho \sin \theta \cos \phi$

$y = \rho \sin \theta \sin \phi$

$z = \rho \cos \theta$

$Y = \rho^2 \sin \theta$

$\rho^2 \sin^2 \theta \cos^2 \phi + \rho^2 \sin^2 \theta \sin^2 \phi + \rho^2 \cos^2 \theta = R^2$

$\int\limits_{0}^{2 \pi} \diff \phi \int\limits_{0}^{\pi/2} \int (\rho^2 \sin^2 \theta \cos^2 \phi + \rho^2 \sin^2 \theta \sin^2 \phi) \rho^2 \sin \theta \diff \rho =
\int\limits_{0}^{2 \pi} \diff \phi \int\limits_{0}^{\pi/2} \int (\rho^2 \sin^2 \theta \cos^2 \phi + \rho^2 \sin^2 \theta \sin^2 \phi) \rho^2 \sin \theta  \diff \rho = \int\limits_{0}^{2\pi} \diff \phi \int\limits_{0}^{\pi/2} \sin^2 \theta \sin \theta \diff \theta \int\limits_{0}^{R} \rho^4 \diff \rho$

$\int\limits_{0}^{\pi/2} (\cos^2 \theta - 1) \diff (\cos \theta) = \frac{\cos^3 \theta}{3} - \cos \theta \bigg|_{0}^{\pi/2} = -\frac{1}{3} + 1 =  \frac{2}{3}$

\pagebreak
\subsection{Применение двойных и тройных интегралов к вычислению объемов}

\subsubsection{№3562}

Найти двойным интегрированием объемы тел, ограниченных данными поверхностями (входящие в условия задач параметры считаются положительными):

Плоскостями $y = 0$, $z = 0$, $3x + y = 6$, $3x + 2y = 12$ и $x + y + z = 6$

Построим график на плоскости: $3x + 2y = 12 \implies y = \frac{12 - 3x}{6}$; $3x + y = 6 \implies y = 6 - 3x$

$V = \int\limits_{0}^{2} \diff x \int\limits_{\frac{6-y}{3}}^{\frac{12-2y}{3}} \diff y \int\limits_{0}^{6-x-y} \diff z + \int\limits_{2}^{4} \diff x \int\limits_{\frac{6-y}{3}}^{\frac{12-2y}{3}} \diff y \int\limits_{0}^{6-x-y} \diff z = \int\limits_{0}^{6} \diff y \int\limits_{\frac{6-y}{3}}^{\frac{12-2y}{3}} \diff x \int\limits_{0}^{6xx-y} \diff z = \dots$

$\int\limits_{\frac{6-y}{3}}^{\frac{12-2y}{3}} (6-x-y) \diff x = (6x - \frac{x^2}{2} - xy) \bigg|_{x=\frac{6-y}{3}}^{x=\frac{12-2y}{3}} = 24 - 4y - \frac{(12-2y)^2}{18} - \frac{12y-2y^2}{3} - 12 + 2y - \frac{(16-y)^2}{18} + \frac{(6y-y^2)}{3} = 12 - 2y-\frac{(12-2y)^2+(6-y)^2}{18} - \frac{12y-2y^2+6y-y^2}{3} = 12-2y-\frac{180-60y+5y^2}{18} - \frac{18y-34}{3} = \frac{216-36y-180+60y-5y^2-108y+18y^2}{18}$

Досчитать предлагается читателю самостоятельно. В аудитории мы таким не занимались.

\subsubsection{№3563}

Найти двойным интегрированием объемы тел, ограниченных данными поверхностями (входящие в условия задач параметры считаются положительными):

Параболоидом вращения $z = x^2 + y^2$, координатными плоскостями и плоскостью $x + y = 1$

Построим график. И на плоскости, и в пространстве.

$V = \int\limits_{0}^{1} \diff x \int\limits_{0}^{1-x} \diff y \int\limits_{0}^{x^2+y^2} \diff z$

$\int\limits_{0}^{x^2 + y^2} \diff z = z \bigg|_{0}^{x^2+y^2} = x^2 + y^2$

$\int\limits_{0}^{1 - x} (x^2+y^2) \diff y = (x^2 y + \frac{y^3}{3}) \bigg|_{y=0}^{y=1-x} = (x^2(1-x) + \frac{(1-x)^3}{3})$

$V = \int\limits_{0}^{1} (x^2(1-x) + \frac{(1-x)^3}{3}) \diff x = \int\limits_{0}^{1} (x^2 - x^3 + \frac{1}{3} (1-3x+3x^2-x^3)) \diff x = \frac{1}{3} - \frac{1}{4} + \frac{1}{3} \int\limits_{0}^{1} (1 - 3x + 3x^2 - x^3) \diff x = \frac{1}{3} - \frac{1}{4} + \frac{1}{3} (1 - \frac{3}{2} + 1 - \frac{1}{4}) = \dots$

Сами посчитайте окончательный результат.

\subsubsection{№3574}

Найти двойным интегрированием объемы тел, ограниченных данными поверхностями (входящие в условия задач параметры считаются положительными):

Цилиндрами $x^2 + y^2 = R^2$, $z = \frac{x^3}{a^2}$ и плоскостью $z = 0$ ($x \ge 0$)

\subsubsection{№3587}

Найти двойным интегрированием объемы тел, ограниченных данными поверхностями (входящие в условия задач параметры считаются положительными):

Цилиндром $x^2 + y^2 = 4$, плоскостями $z = 0$ и $z = x + y + 10$

\subsection{Домашнее задание}

№3565, №3588, №3609, №3610, №3618

\end{document}